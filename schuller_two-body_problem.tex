% \DocumentMetadata{pdfstandard=A-3b}
\documentclass[12pt, a4 paper]{article}

\usepackage{embedall}
\embedfile{schuller_two-body_problem.bib}

\usepackage[ngerman,english]{babel}
\PassOptionsToPackage{math-style=ISO, bold-style=ISO, sans-style=italic, nabla=upright, partial=upright, warnings-off={mathtools-colon,mathtools-overbracket}}{unicode-math}
% \PassOptionsToPackage{warnings-off={mathtools-colon,mathtools-overbracket}}{unicode-math}
\usepackage{mathtools, microtype}
\usepackage{setspace}
\usepackage[style=british]{csquotes}
% \usepackage{unicode-math}
% \usepackage{fontsetup}
% \usepackage[mathscr]{eucal}
% \setmathfontface\itcal{NewCMMath-Book}[FakeSlant=-0.25]
% \usepackage[math-style=ISO, bold-style=ISO, sans-style=italic, nabla=upright, partial=upright, warnings-off={mathtools-colon,mathtools-overbracket}]{unicode-math}
\usepackage{kpfonts-otf}
\setmathfont{KpMath-Regular}
\setmathfont{KpMath-Regular}[range={cal,bfcal},StylisticSet=1]
% \setmathfont{KpMath-Regular}[range={cal,bfcal},StylisticSet=1]
% \setmainfont[Ligatures=Required]{KpRoman}
% \setmathfont{KpMath-Bold.otf}[version=bold]
% \setmathfont{KpMath-Regular.otf}
% \setmathfont{KpMath-Regular.otf}[range={cal,bfcal},StylisticSet=,BoldFont=KpMath-Bold.otf]
% \setmathfont{KpMath-Regular.otf}[version=regular]
% \setmathfont{KpMath-Regular}[range={cal,bfcal},StylisticSet=1]
% \usepackage{sectsty}
% \allsectionsfont{\mathversion{bold}}
% \setmathfontface\itcal{KpMath-Regular.otf}[FakeSlant=-0.4]
% \usepackage{newtx}
% \usepackage[bold=0.1]{xfakebold}
% \usepackage{xfp}
% \usepackage{dsfont}
% \renewcommand{\mathbb}{\mathbf}
% \renewcommand{\mathbb}[1]{\setBold \mathds #1 \unsetBold}
% \newcommand{\bmathscr}[1]{\setBold \mathscr #1 \unsetBold}
% \renewcommand{\symbb}{\mathds}
% \renewcommand{\symbb}{\symbfup}
% \usepackage[T1]{fontenc}		% use this if European fonts (ec) package is installed
% \usepackage{textcomp}			% adds some additional symbols
% \usepackage[scaled=0.92]{helvet} % set Helvetica as the sans-serif font
% \renewcommand{\rmdefault}{ptm} % set Times as the default text font
% \usepackage{newtxtext}
% \usepackage{fontspec}
% \setmainfont{Times New Roman}
% \setsansfont{Helvetica}
% \setmonofont{Courier}
% \usepackage[lite, subscriptcorrection, compatiblegreek]{mtpro2}

% \usepackage{mlmodern}
\usepackage{amsthm}
\theoremstyle{definition}
\newtheorem{thm}{Theorem}
\newtheorem{defn}{Definition}

\title{The Quantum Mechanical Two-body Problem}
% \author{Frederic Schuller\\Apoorv Potnis}
\author{}
\date{\vspace{-5ex}}

\usepackage[sorting=none]{biblatex}
\addbibresource{schuller_two-body_problem.bib}
\usepackage{hyperref}
\hypersetup{linktoc=all, citecolor=red, pdfencoding=auto, psdextra, colorlinks=true, linkcolor=red, breaklinks=true, urlcolor=blue, pdftitle={Schuller – The Quantum Mechanical Two-body Problem}, bookmarksopen=true, pdfauthor={Apoorv Potnis}, pdfsubject={Two-body Problem – Quantum Mechanics}, unicode=true, pdftoolbar=true, pdfmenubar=true, pdfstartview={FitH}, pdfkeywords={Frederic Schuller, Hydrogen Atom, Quantum Mechanics, Mathematical Physics, Lecture Notes, Angular Momentum, Two-body Problem, Kepler problem}}
% \hypersetup{linktoc=all, citecolor=black, pdfencoding=auto, psdextra, colorlinks=true, linkcolor=black, breaklinks=true, urlcolor=black, pdftitle={Schuller – Rigged Hilbert Spaces}, bookmarksopen=true, pdfauthor={Apoorv Potnis}, pdfsubject={Rigged Hilbert Spaces – Quantum Mechanics}, unicode=true, pdftoolbar=true, pdfmenubar=true, pdfstartview={FitH}, pdfkeywords={Frederic Schuller, Rigged Hilbert Spaces, Quantum Mechanics, Mathematical Physics, Lecture Notes, Bras, Kets}}
\usepackage{caption, cleveref, xurl, bookmark}

% % \newcommand{\ltwo}{\mathup{L\kern-0.5pt^2}}
% \newcommand{\position}{\mathup{Q}}
% \newcommand{\momentum}{\mathup{P}}
% \newcommand{\rthree}{\mathbb{R}^3}
% \newcommand{\rr}{\mathbb{R}}
% \newcommand{\cc}{\mathbb{C}}
% \newcommand{\nn}{\mathbb{N}_0}
% \newcommand{\dirac}{\symup{\delta}}
% % \newcommand{\hilbert}{\itcal{ℋ}\kern-0.7pt}
% \newcommand{\hilbert}{\mathcal{H}}
% \newcommand{\ltwor}{\ltwo(\rr)}
% % \newcommand{\poly}{\itcal{𝒫}\kern-0.5pt}
% \newcommand{\poly}{𝒫}
% \newcommand{\schwartz}{\kern0.5pt\itcal{𝒮}\kern-2pt}
% % \newcommand{\schwartz}{\mathcal{S}}
% \newcommand{\schwartzr}{\schwartz(\rr)}
% \newcommand{\dist}{\itcal{\schwartz^\times}\kern-0.5pt}
% % \newcommand{\dist}{\schwartz^\times}
% \newcommand{\distr}{\dist(\rr)}
% \newcommand{\distar}{\schwartz^{\kern-0.1pt*}(\rr)}
% \newcommand{\anti}{\overline{\distr}}
% \renewcommand*{\hbar}{\mathrm{^^^^0127}}
% \renewcommand{\i}{\mathrm{i}}
% \newcommand{\e}{\mathrm{e}}
% \newcommand{\cinfinity}{\mathrm{C}^\infty}
% \newcommand{\family}{\mathcal{F}}
% \newcommand{\lift}{\hat{D}}
% \newcommand{\domain}{\mathcal{D}}
% \newcommand{\identity}{\mathrm{id}}
% \newcommand{\w}{w\kern-1pt}
% \DeclareMathOperator{\spec}{spec}
% \newcommand{\resolvent}{\mathup{\rho}}

\newcommand{\ltwo}{\mathup{L\kern-0.7pt^2}}
\newcommand{\ltworthree}{\ltwo(\rr^3)}
\newcommand{\ltworsix}{\ltwo(\rr^6)}
\newcommand{\position}{Q}
\newcommand{\momentum}{P}
\newcommand{\rthree}{\mathbb{R}^3}
\newcommand{\rr}{\mathbb{R}}
\newcommand{\cc}{\mathbb{C}}
\newcommand{\nn}{\mathbb{N}_0}
\newcommand{\dirac}{\delta}
% \newcommand{\hilbert}{\itcal{ℋ}\kern-0.7pt}
\newcommand{\hilbert}{\mathcal{H}}
\newcommand{\ltwor}{\ltwo(\rr)}
% \newcommand{\poly}{\itcal{𝒫}\kern-0.5pt}
\newcommand{\poly}{\mathcal{P}}
\newcommand{\schwartz}{\symcal{S}}
% \newcommand{\schwartz}{\mathcal{S}}
\newcommand{\schwartzr}{\schwartz(\rr)}
\newcommand{\schwartzrthree}{\schwartz(\rr^3)}
\newcommand{\dist}{\schwartz^\times}
% \newcommand{\dist}{\schwartz^\times}
\newcommand{\distr}{\dist(\rr)}
\newcommand{\distar}{\schwartz^*(\rr)}
\newcommand{\anti}{\overline{\distr}}
% \renewcommand*{\hbar}{\mathrm{^^^^0127}}
\renewcommand{\i}{\mathrm{i}}
\newcommand{\e}{\mathrm{e}}
\renewcommand{\pi}{\muppi}
\newcommand{\cinfinity}{\mathrm{C}^\infty}
\newcommand{\family}{\mathcal{F}}
\newcommand{\lift}{\hat{D}}
\newcommand{\domain}{\mathcal{D}}
\newcommand{\identity}{\mathrm{id}}
\newcommand{\w}{w\kern-1pt}
\DeclareMathOperator{\spec}{spec}
\newcommand{\resolvent}{\mathup{\rho}}
\newcommand{\lap}{∆}
% \renewcommand{\symbfit}[1]{\symbfit{#1}}
\newcommand{\xone}{\symbfit{x}^{(1)}}
\newcommand{\xtwo}{\symbfit{x}^{(2)}}
\newcommand{\rvec}{\symbfit{r}}
\newcommand{\yvec}{\symbfit{y}}
\newcommand{\angsone}{\tilde{L}_1}
\newcommand{\angstwo}{\tilde{L}_2}
\newcommand{\angsthree}{\tilde{L}_3}
\newcommand{\lvecsquare}{\tilde{\symbfit{L}}\kern-1pt^2}
\newcommand{\parr}{\frac{\partial}{\partial r}}
\newcommand{\partheta}{\frac{\partial}{\partial \theta}}
\newcommand{\parphi}{\frac{\partial}{\partial \phi}}
\newcommand{\levi}{\epsilon_{ijk}}

\DeclarePairedDelimiter{\norm}{\lVert}{\rVert}
\DeclarePairedDelimiter{\abs}{\lvert}{\rvert}
\DeclarePairedDelimiterX\set[1]\lbrace\rbrace{\def\given{\;\delimsize\vert\;}#1}
\newcommand{\der}{\operatorname{d\!}{}}
\usepackage{mleftright}

\usepackage{luatex85}
\newsavebox{\foobox}
\newcommand{\slantbox}[2][.5]
{%
	\mbox
	{%
		\sbox{\foobox}{#2}%
		\hskip\wd\foobox
		\pdfsave
		\pdfsetmatrix{1 0 #1 1}%
		\llap{\usebox{\foobox}}%
		\pdfrestore
	}%
}

\usepackage{etoolbox}
\apptocmd{\sloppy}{\hbadness 10000\relax}{}{}
\emergencystretch=1em

\makeatletter
\g@addto@macro\bfseries{\boldmath}
\makeatother

\usepackage{tikz, tkz-euclide, pgfplots}
\pgfplotsset{compat=1.18}
\setmathfont{KpMath-Sans.otf}[version=sans]
\setmathfont{KpMath-Regular.otf}[version=base, range={cal,bfcal},StylisticSet=1]
\usepackage{tikz-cd}
\tikzcdset{arrow style=math font}
\usetikzlibrary{decorations.markings}
\usetikzlibrary {arrows.meta,bending}
% \setmathfont{KpMath-Regular_arrow.otf}
% \tikzset{
% 	-math/.style={
% 		shorten >=1.4\pgflinewidth,
% 		postaction=decorate,
% 		decoration={markings, mark=at position 0.89 with {
% 				\node[scale=1.4\pgflinewidth, xshift=-1.7pt,transform shape,rotate=-91]{$\char"2191$}; }}}
% }
% \usepackage{fontsetup}
% \setmathfont{KpMath-Regular_arrow.otf}

\usepackage{tikz-3dplot}

\tdplotsetmaincoords{60}{110}
%
\pgfmathsetmacro{\rlength}{.8}
\pgfmathsetmacro{\thetavec}{30}
\pgfmathsetmacro{\phivec}{60}

\begin{document}

	\maketitle

	These are lecture notes by Apoorv Potnis of the lecture `\selectlanguage{ngerman}Quantenmechanisches Zweikörperproblem\selectlanguage{english}' or `The Quantum Mechanical Two-body Problem', given by \textbf{Prof.\@ Frederic Paul Schuller}, as the seventeenth lecture in the course `\selectlanguage{ngerman}Theoretische Physik 2: Theoretische Quantenmechanik\selectlanguage{english}' in 2014/15 at the \selectlanguage{ngerman}Friedrich-Alexander-Universität Erlangen-Nürnberg\selectlanguage{english}. While the original lecture is in German, these notes are in English and have been prepared using YouTube's automatic subtitle translation tool. The lecture is available at \url{https://www.youtube.com/watch?v=mcM4S3IMMvI&list=PLPO5pgr_frzTeqa_thbltYjyw8F9ehw7v&index=17} and at \url{https://www.fau.tv/clip/id/44891}.

	\tableofcontents

	\section{Introduction}

	\begin{figure}
		\mathversion{sans}
	    \centering
		\begin{tikzpicture}
% 			\node (mone) at (0,0) {\(m_1\)};
			\filldraw [black] (0,0) circle (6pt);
			\node (mone) at (0.4,0.4) {\(m_1\)};
			\filldraw [black] (1.6,-1.6) circle (6pt);
			\node (mtwo) at (1.2,-2) {\(m_2\)};
			\draw [black, dashed, thick] (0,0) -- (1.6,-1.6) node [midway, above right] {\(r\)};
			\tkzDefPoint(0.8,-0.8){O}
			\tkzDrawArc[R, color=black, line width=0.6pt, -{Straight Barb[bend]}](O,1.1cm)(340,10)
			\tkzDrawArc[R, color=black, line width=0.6pt, -{Straight Barb[bend]}](O,1.1cm)(160,190)

% 			\draw [black, thick, ->] (0,0) arc [start angle=10, end angle=40, x radius=1.6cm, y radius =1.6cm];

% 			\draw[black, thick, ->, xshift=0.8cm, yshift=-0.8cm] (-30:0.8cm) arc (-30:-10:0.8cm);
		\end{tikzpicture}
		\caption*{\textsf{Classical picture}}\label{fig:twobody}
		\mathversion{base}
	\end{figure}
	Consider a classical system system consisting of two interacting particles of masses \(m_1\) and \(m_2\), such that the interaction is completely determined by the potential $V(r)$, and the potential depends only on the distance $r$ between the particles. If we consider the central force problem, like a planet orbiting the sun, then we know we have bound states corresponding to \textit{all} the negative energy values, and scattering states corresponding to all the positive energy values. It turns out that if consider the system quantum-mechanically, then bound states cannot admit all the negative energy solutions. Only certain energy levels are allowed. This can be seen experimentally from the discrete lines in the spectra of atoms, demonstrating the quantised nature of energy levels. In this lecture, we shall consider the quantum-mechanical case of the two-body problem.

	\begin{figure}
	    \centering
		\mathversion{sans}
		\begin{tikzpicture}[line cap=round]
			%		%Grid
			%		\draw[thin, dotted] (0,0) grid (8,8);
			%		\foreach \i in {1,...,8}
			%		{
			%			\node at (\i,-2ex) {\i};
			%		}
			%		\foreach \i in {1,...,8}
			%		{
			%			\node at (-2ex,\i) {\i};
			%		}
			%		\node at (-2ex,-2ex) {0};

% 			% Axis
% 			\draw[thick, -{Straight Barb[bend]}] (0,0) -- (5,0) node[above] {$r$};
% 			\draw[thick, {Straight Barb[bend]}-{Straight Barb[bend]}] (0,-4.5) -- (0,0.5) node[right] {$V(r)$};
% 			\node at (-0.3,0) {$0$};

			\begin{axis}[
				samples=500,
				domain=0.24:4.5,
				xmin=0, xmax=5,
				ymin=-4.4, ymax=0.5,
				axis lines=middle,
				ticks=none,
				width=0.5\textwidth,
% 				height=5cm,
				xlabel={$r$},
				ylabel={$V(r)$},
				x axis line style={thick, -{Straight Barb[bend]}},
				y axis line style={thick, {Straight Barb[bend]}-{Straight Barb[bend]}},
				y label style={anchor=south},
				x label style={anchor=west},
				legend style={draw=none, at={(axis cs:2,-1.8)},anchor=north west},
				legend style={row sep=10pt}
				]
				\addplot[thick, red] {-exp(-\x)/(\x)};
				\addplot[thick, blue, dashed] {-1/(\x)};
				\addlegendentry{$\displaystyle-\frac{\e^{-r}}{r}$};
				\addlegendentry{$\displaystyle-\frac{1}{r}$};
			\end{axis};


			% Plot Function
	% 		\draw[domain=0.177:7.5, samples=600, variable=\r, very thick] plot ({\r},{0.3/(\r*\r)-0.9/\r});
	% 		\draw[domain=0.6:7.5, samples=500, variable=\r, thick, dashed, red] plot ({\r},{1/(\r*\r)});
% 			\draw[domain=0.24:4.5, samples=500, variable=\r, thick, blue] plot ({\r},{-1/(\r)});
% 			\draw[domain=0.24:4.5, samples=500, variable=\r, thick, red] plot ({\r},{-exp(-\r)/(\r)});

			% Dashed
% 			\draw[dashed] (2/3,0) -- +(0,-0.65) node[pos=0, above] {$r_o$};
% 			\draw[dashed] (2/3,-0.68) -- (0,-0.68);

			% Nodes
% 			\node at (0.6,3.5) {$V_\text{eff}$};
% 			\node[red] at (1.3,1.5) {$1/r^2$};
% 			\node[blue] at (1.4,-1.5) {$-1/r$};
		\end{tikzpicture}
		\caption*{\textsf{Yukawa and Coulomb potentials}}\label{fig:potential}
		\mathversion{base}
	\end{figure}
	An example of a potential as described above would be the \textit{Yukawa potential}, defined as
	\begin{align*}
		V_{\text{Yukawa}}(r) \coloneq a \frac{\exp({-kmr})}{r},
	\end{align*}
	where $k$, $m$ and $a$ are constants. $a \in \rr \setminus \{0\}$, $m \geq 0$. According to quantum field theory, very roughly speaking, interaction between particles takes place via a `mediating particle'. If the interaction is mediated by a `scalar field'\footnote{Whatever that means}, then the mass associated to the particle of that scalar field is the mass $m$ appearing in the Yukawa potential. If we plot a graph of the Yukawa potential for a massive scalar field, then we see that the magnitude of the potential becomes very close to zero after a certain distance. Thus, these interactions are short-ranged. If instead we have $m = 0$, corresponding to a photon, then we get the familiar long-range Coulomb potential
	\begin{align*}
		V_{\text{Coulomb}}(r) \coloneq a \frac{1}{r}.
	\end{align*}

	We also have the finite wall potential $V_{\text{finite wall}}(r) \coloneq a \upTheta(r - r_0)$ and the isotropic harmonic oscillator $V_{\text{ihc}} (r) \coloneq a r^2$.

	The Hilbert space of the individual particles is \(\ltworthree\) and the Hilbert space of the composite two-body system is \(\ltworthree \otimes \ltworthree\), which is naturally isomorphic to \(\ltworsix\). This can be seen from theorem II.10 of the book of Reed and Simon~\cite[p.~52]{Reed}. If $f, g \in \ltworthree$, then we can define an isomorphism \(\ltworthree \otimes \ltworthree \rightarrow \ltworsix\) by \(f \otimes g \mapsto fg \in \ltworsix\).

	The Hamiltonian for our quantum-mechanical two-body system is
	\begin{align*}
		H \coloneq -\frac{\hbar^2}{2m_1}\lap_{(1)} - \frac{\hbar^2}{2m_2}\lap_{(2)} + V\left(\norm{\xone - \xtwo}\right),
	\end{align*}
	where \(\lap_{(i)}\) is the Laplacian operator which acts on the Hilbert space \(\ltworthree\) of the particle \(i\), and \(\symbfit{x}^{(i)} \in \rthree\) denotes the position vector of the \(i^\text{th}\) particle. We are interested in finding out the spectrum of this Hamiltonian.

	As is the case in classical mechanics, we shall move to the center-of-mass co-ordinates. We can move the center-of-mass of the system in any direction by any value, i.e. the spectrum of the `center-of-mass' position operator is \(\rr\). The discrete energy levels are due to the angular momentum about the center-of-mass. We thus introduce
	\begin{align*}
		\yvec &\coloneq \frac{m_1 \xone + m_2 \xtwo}{m_1 + m_2},\\
	    \rvec &\coloneq \xone - \xtwo,\\
		\mu &\coloneq \frac{m_1 m_2}{m_1 + m_2}.
	\end{align*}
	The Hamiltonian then takes the form
	\begin{align*}
		H \coloneq -\frac{\hbar^2}{2(m_1 + m_2)}\lap_{\yvec} - \frac{\hbar^2}{2\mu}\lap_{\rvec} + V(\norm{\rvec}).
	\end{align*}
	The wave function of the system \(\psi\left(\xone, \xtwo\right)\) expressed in the new co-ordinates becomes
	\begin{align*}
	    \phi(\yvec, \rvec) \coloneq \psi\left(\xone(\yvec, \rvec), \xtwo(\yvec,\rvec)\right).
	\end{align*}
	The Hamiltonian $H$ can be separated into two parts, $H_\text{free}$ describing the movement of the center-of-mass and $H_\text{rel}$, describing the movement of the particles around the center-of-mass. We have
	\begin{align*}
		H_\text{free} \coloneq -\frac{\hbar^2}{2(m_1 + m_2)}\lap_{\yvec} \qquad \text{ and } \qquad H_\text{rel} \coloneq - \frac{\hbar^2}{2\mu}\lap_{\rvec} + V(\norm{\rvec}).
	\end{align*}
	We know that $\spec{(H_\text{free})} = \rr_{\geq 0}$, so the task that remains is to find $\spec{(H_\text{rel})}$. We see that $H_\text{rel}$ is spherically symmetric. Thus, we shall move to the spherical co-ordinates to separate the angular and radial dependencies. This shall reduce our three-dimensional problem to a one-dimensional one.

	\section{Moving to spherical co-ordinates}

	\begin{figure}
		\centering\mathversion{sans}
		\begin{tikzpicture}[scale=4,tdplot_main_coords]
			\coordinate (O) at (0,0,0);
			\draw[thick,-Straight Barb] (0,0,0) -- (0.7,0,0) node[anchor=north east]{$x$};
			\draw[thick,-Straight Barb] (0,0,0) -- (0,0.7,0) node[anchor=north west]{$y$};
			\draw[thick,-Straight Barb] (0,0,0) -- (0,0,0.7) node[anchor=south]{$z$};
			\tdplotsetcoord{P}{\rlength}{\thetavec}{\phivec}

			\draw[dashed, thick] (O) -- (Pxy);
			\draw[dashed, thick] (P) -- (Pxy);
			\tdplotdrawarc[thick]{(O)}{0.2}{0}{\phivec}{anchor=north}{$\phi$}
			\tdplotsetthetaplanecoords{\phivec}
			\tdplotdrawarc[thick, tdplot_rotated_coords]{(0,0,0)}{0.5}{0}%
			{\thetavec}{anchor=south west}{$\theta$}
			\draw[-Straight Barb,color=red, thick] (O) -- (P) node[above right] {$\rvec$};
			\node (r) at (0,0.15,0.2) {\(r\)};
		\end{tikzpicture}
		\caption*{\textsf{Spherical co-ordinates}}\label{fig:sphericalcoordinates}
		\mathversion{base}
	\end{figure}

	Now we know that the angular momentum operators are self-adjoint and defined on their Stone domains. We shall instead consider the restrictions of the angular momentum operators on the Schwartz space to avoid running into technical issues. The Schwartz space is dense in the Stone domains and $\ltworthree$. These restrictions are essentially self-adjoint. We shall now on not distinguish between the angular momentum operators and their essentially self-adjoint restrictions on the Schwartz space. Let $i \in \{1,2,3\}$. We thus have the Cartesian angular momentum operators
	\begin{align*}
	    L_i &\colon \schwartzrthree \rightarrow \schwartzrthree,\\
		L_i &\colon \psi \mapsto \levi\position_j\momentum_k\psi,
	\end{align*}
	where $\position_j$ and $\momentum_k$ denote the $j^\text{th}$ and $k^\text{th}$ position and momentum operators respectively. The indices run over 1, 2 and 3.
	We thus have
	\begin{align*}
	    L_1 &= -\i\hbar(y\partial_z - z\partial_y)\\
		L_2 &= -\i\hbar(z\partial_x - x\partial_z)\\
		L_3 &= -\i\hbar(z\partial_y - y\partial_x).
	\end{align*}
	Let
	\begin{align*}
	    A \coloneq [0, +\infty) \times [0,\pi] \times [0,2\pi).
	\end{align*}
	Let $T$ be defined as
	\begin{align*}
		T &\colon A \rightarrow \rthree,\\
		T &\colon (r, \theta, \phi) \mapsto (x, y, z),
	\end{align*}
	where
	\begin{align*}
		x &= r \cos{\phi} \sin{\theta},\\
		y &= r \sin{\phi} \sin{\theta},\\
		z &= r \cos{\theta}.
	\end{align*}
	Thus,
	\begin{align*}
	    T(r, \theta, \phi) \coloneq (r \cos{\phi} \sin{\theta}, \sin{\phi} \sin{\theta}, \cos{\theta}).
	\end{align*}
	If $F \in \schwartz(A)$ corresponds to $\psi$ in spherical co-ordinates\footnote{\,Note that even though $T$ is only surjective and not injective, $T^{-1}(\psi)$ is a measure zero set for any $\psi \in \schwartzrthree$. Thus, we don't need to make any further identifications in $\schwartz(A)$.}, we have
	\begin{align*}
		F = \psi \circ T,
	\end{align*}
	and the chain rule gives us
	\begin{align*}
	    F'(r, \theta, \phi) = \psi'(x,y,z) \cdot T'(r, \theta, \phi),
	\end{align*}
	where $(x,y,z) = T(r, \theta, \phi)$. We thus have
	\begin{align*}
	    \partial_\phi = (\partial_\phi x)\partial_x + (\partial_\phi y)\partial_y + (\partial_\phi z)\partial_z.
	\end{align*}
	Substituting $x = r \cos{\phi} \sin{\theta}$, $y = r \sin{\phi} \sin{\theta}$ and $z = r \cos{\theta}$, we get
	\begin{align*}
		\partial_\phi &= -(r \sin{\theta} \sin{\phi})\partial_x + (r \sin{\theta} \cos{\phi})\partial_y\\
		\partial_\phi &= x\partial_2 - y\partial_1.
	\end{align*}
	Thus, as expected on physical grounds, we get
	\begin{align*}
		\tilde{L}_3 = -\i\hbar\partial_\phi,
	\end{align*}
	where $\tilde{L}_3 \colon \schwartz(A) \rightarrow \schwartz(A)$ denotes the angular momentum operator in spherical co-ordinates.
	We now restrict the domain of $T$ suitably such that it becomes injective and $T^{-1}$ becomes a well defined function. More precisely, we take the domain to be $\tilde{A} \coloneq (0, +\infty) \times (0,\pi) \times (0,2\pi)$. This does not cause any problems as we are excluding only measure zero sets which are irrelevant when considering wave-functions. If we define $\tilde{\rthree}$ as $\rthree$ excluding the axes, we have $T^{-1} \colon \tilde{\rthree} \rightarrow \tilde{A}$ as
	\begin{align*}
		r      &= \sqrt{x^2 + y^2 + z^2},\\
		\theta &= \cos^{-1}{\left(\frac{z}{\sqrt{x^2 + y^2 + z^2}}\right)},\\
		\phi   &= \cos^{-1}{\left(\frac{x}{\sqrt{x^2 + y^2}}\right)}.
	\end{align*}
	Applying chain rule to $\psi = F \circ T^{-1}$ gives us
	\begin{align*}
	    \partial_x &= (\partial_x r)\partial_r + (\partial_x \theta)\partial_\theta + (\partial_x \phi)\partial_\phi,\\
		\partial_y &= (\partial_y r)\partial_r + (\partial_y \theta)\partial_\theta + (\partial_y \phi)\partial_\phi,\\
		\partial_z &= (\partial_z r)\partial_r + (\partial_z \theta)\partial_\theta + (\partial_z \phi)\partial_\phi.
	\end{align*}
	Calculating the partial derivatives and substituting the above in $L_1 = -\i\hbar(y\partial_z - z\partial_y)$ and $L_2 = -\i\hbar(z\partial_x - x\partial_z)$ finally gives us
	\begin{align*}
		\angsone &\coloneq \i\hbar\left(\cos{\phi}\cot{\theta}\frac{\partial}{\partial\phi} + \sin{\phi}\frac{\partial}{\partial\theta}\right),\\
		\angstwo &\coloneq \i\hbar\left(-\sin{\phi}\cot{\theta}\frac{\partial}{\partial\phi} + \cos{\phi}\frac{\partial}{\partial\theta}\right),\\
		\angsthree &\coloneq -\i\hbar \frac{\partial}{\partial\phi}.
	\end{align*}
	We also have
	\begin{align*}
		\lvecsquare F &= -\frac{1}{\sin{\theta}}\frac{\partial}{\partial \theta}\left(\sin{\theta}\frac{\partial}{\partial\theta}F\right) -\frac{1}{\sin{\theta}}\frac{\partial}{\partial\phi}\left(\frac{1}{\sin{\theta}}\frac{\partial}{\partial\theta}F\right),
	\end{align*}
	and
	\begin{align*}
		\tilde{H}_\text{rel} F = -\frac{\hbar^2}{2\mu}\left(\frac{1}{r}\left(\frac{\partial}{\partial r}\right)^2 (rF) - \frac{\lvecsquare}{r^2}F\right) + V(r)F.
	\end{align*}
	In order to fully determine $\spec(H_\text{rel})$, along with the continuous spectrum, i.e.\@ the part containing the generalised eigenvalues, we would need to lift the eigenvalue equation to the distribution space and solve it.\footnote{\,This can be achieved using Gelfand's rigged Hilbert space formalism, which was introduced in an earlier lecture. The lecture itself can be found on this link: \url{https://www.youtube.com/watch?v=FNJOyxOp3Ik&list=PLPO5pgr_frzTeqa_thbltYjyw8F9ehw7v&index=8}. The notes for this lecture can be found here: \url{https://github.com/apoorvpotnis/schuller_rigged_hilbert_spaces/blob/main/schuller_rigged_hilbert_spaces.pdf}.} However, we already know from experiments that the hydrogen atom has discrete lines in its observed spectrum. Thus, we shall first solve for the point spectrum
	\begin{align*}
		\spec_{\text{p}}(H_\text{rel}) = \set{E \in \rr \given H_\text{rel} F = E F},
	\end{align*}
	which was historically a major validation for the quantum theory.

	\section{Spherical Harmonics}

	In the previous lectures, we have already discussed the simultaneous eigenvectors and eigenvalues of $\symbfit{L}\kern-1pt^2$ and $L_a$, as they commute with each other. We thus consider the simultaneous eigenvalues of $H_\text{rel}$, $\symbfit{L}\kern-1pt^2$ and $L_3$ in the hope that this will give us more equations to work with and make the job easier. It can be seen that all three operators commute with each other pairwise. Up to isomorphism, the eigenvectors and eigenvalues of an operator are not affected by co-ordinate changes, as $\schwartzrthree$ and $\schwartz(\tilde{A})$ are isomorphic to each other.

	We know that the common eigenvectors of $\lvecsquare$ and $\tilde{L}_3$ come as families $\psi\kern0.5pt_l^m \in \schwartz(\tilde{A})$, where $m = -l, -l+1, \ldots, l-1, l$ and $l \in \nn$. We have
	\begin{align*}
	    \lvecsquare \psi\kern0.5pt_l^m &= l(l+1)\psi\kern0.5pt_l^m,\\
		\tilde{L}_3 \psi\kern0.5pt_l^m &= m \psi\kern0.5pt_l^m.
	\end{align*}
	Note that $l$ can take only integer values, not half-integer values as we are dealing with the orbital angular momentum operator. Also recall the raising and lowering operators $\tilde{L}_+ \coloneq \tilde{L}_1 + i \tilde{L}_2$ and $\tilde{L}_- \coloneq \tilde{L}_1 - i\tilde{L}_2$. We have
	\begin{align*}
		\tilde{L}_\pm \psi\kern0.5pt_l^m = \sqrt{l(l+1) - m(m\pm 1)} \psi\kern0.5pt_l^m.
	\end{align*}
	In spherical co-ordinates, it turns out that the solutions $\psi\kern0.5pt_l^m$ are given by
	\begin{align*}
		\psi\kern0.5pt_l^m (\tilde{\symbfit{x}} (r, \theta, \phi)) = k \cdot Y^m_l (\theta, \phi) \cdot f(r),
	\end{align*}
	where
	\begin{align*}
		Y^m_l (\theta, \phi) \coloneq \frac{(-1)^m}{2^l l!}\left(\frac{(2l+1) (l-m)!}{4\pi (l+m)!}\right)^{\frac{1}{2}}\!\cdot\e^{\i m\phi}\cdot\!(\sin{\theta})^m \left(\frac{\partial}{\partial \cos{\theta}}\right)^{l+m}\!\!\!(\cos^2{\theta} -1)^l
	\end{align*}
	and $\tilde{\symbfit{x}} = T(\symbfit{x})$. These $Y_l^m$'s are called as \textit{spherical harmonic functions}. Since the operators $\lvecsquare$ and $\tilde{L}_3$ act only on the angular components and not the radial components when expressed in spherical co-ordinates, we have that
	\begin{align*}
	    \lvecsquare Y_l^m &= l(l+1) Y_l^m,\\
		\tilde{L}_3 Y_l^m &= m Y_l^m,
	\end{align*}
	i.e.\@ the spherical harmonic functions are eigenvectors of $\lvecsquare$ and $\tilde{L}_3$. These functions actually form an orthonormal basis of the Hilbert space $\mathrm{L}\kern-0.7pt^2(\mathbb{S}^2)$ consisting of complex square-integrable functions defined on the 2-sphere.
	\begin{align*}
		\langle Y_l^m , Y_{l'}^{m'} \rangle = \delta\kern0.5pt_{l, l'}\cdot\delta\kern0.4pt_{m, m'}.
	\end{align*}
	We request that the reader consult section 14.2 of the book of Bowers for a derivation of these spherical harmonic functions~\cite[p.~200]{Bowers}.

	\section{Spectrum of the Hamiltonian operator}

% 	\section{Angular momentum in spherical co-ordinates}\label{angmomentumsph}














%

	\nocite{*}
	\printbibliography[heading=bibintoc]

	\par\begin{spacing}{0.6}
		{\footnotesize The source code, updates and corrections to this document can be found on this GitHub repository: \url{https://github.com/apoorvpotnis/schuller_two-body_problem}. The source code, along with the \texttt{.bib} file is embedded in this PDF. Comments and corrections can be mailed at \href{mailto:apoorvpotnis@gmail.com}{\texttt{apoorvpotnis@gmail.com}}.}
	\end{spacing}

\end{document}

\begin{align*}
	(\partial_\phi F)(r, \theta, \phi) &= \partial_x (\psi(x, y, z)) \partial_\phi(T(r, \theta, \phi))\\
	&= \partial_x (\psi(x, y, z)) \partial_\phi(r\cos{\phi}\sin{\theta})\\
\end{align*}




Let $D_r$ denote the partial derivative operator with respect to $\phi$.
\begin{align*}
	D_r &\colon \schwartz(A) \rightarrow \schwartz(A),\\
	D_r &\colon F \mapsto D_rF.
\end{align*}
We similarly define $D_\phi$ and $D_\theta$. Let $D_i$ denote the partial derivative operator with respect to $i \in {1, 2, 3}$.
\begin{align*}
	D_i &\colon \schwartzrthree \rightarrow \schwartzrthree\\
	D_i &\colon \psi \mapsto D_i\psi.
\end{align*}
Expressing the chain rule in terms of partial derivatives, we have







Let $\tilde{T}$ denote the operator
\begin{align*}
	\tilde{T} \schwartz(A) \
\end{align*}





Now, if $F \in \schwartz(A)$, then we have the co-ordinate transformation function $B$ as
\begin{align*}
	B &\colon \schwartz(A) \rightarrow \schwartzrthree,\\
	B &\colon F \mapsto BF = \psi,
\end{align*}
given by
\begin{align*}
	(BF)(x, y, z) = \psi(x,y,z) \coloneq F\bigg(\sqrt{x^2 + y^2 + z^2}, \frac{x}{\sqrt{x^2 + y^2}}, \frac{z}{\sqrt{x^2 + y^2 + z^2}}\bigg),
\end{align*}
when $\sqrt{x^2 + y^2 + z^2}$ and $\sqrt{x^2 + y^2}$ do not vanish. When they vanish, we define then using their limits, which vanish identically.




We consider the spherical angular momentum operators
\begin{align*}
	\tilde{L_i} &\colon \schwartz(A) \rightarrow \schwartz(A),\\
	\tilde{L_i} &\colon F \rightarrow \tilde{L_i}F,
\end{align*}
given by
\begin{align*}
	\tilde{L_i}F \coloneq L_iBF = L_i\psi.
\end{align*}
Since $\tilde{L_i} = L_iB$, we have
\begin{align*}
	\tilde{L_i} = -\i\hbar\levi\position_jD_kB,
\end{align*}
where $D_k$ denotes the $k^\text{th}$ partial derivative.
Consider $L_3$, i.e.\@ let $i = 3$. We have









\begin{align*}
	L_i
\end{align*}

























Expressed in spherical co-ordinates, we have the wave-function as
\begin{align*}
	F(r, \theta, \phi) \coloneq \psi(\rvec(r, \theta, \phi)),
\end{align*}
where
\begin{align*}
	r_1 &= r \cos{\phi} \sin{\theta},\\
	r_2 &= r \sin{\phi} \sin{\theta},\\
	r_3 &= r \cos{\theta},
\end{align*}
and $r_1$, $r_2$ and $r_3$ denote the $x$-, $y$- and $z$-components of $\rvec$ respectively. We now express the angular momentum operators in spherical co-ordinates as well. If $L_a$ denotes $a^{\text{th}}$ ($a \in \{1,2,3\}$) angular momentum operator in Cartesian co-ordinates, then we have
\begin{align*}
	(\tilde{L}_a F) (r, \theta, \phi) = (L_a \psi)(\rvec(r, \theta, \phi)),
\end{align*}
where $\tilde{L}_a$ denotes the angular momentum operator in spherical co-ordinates. We have
\begin{align*}
	L_a &\colon \ltworthree \rightarrow \ltworthree,\\
	L_a &\colon \psi(\rvec) = \psi(r_1, r_2, r_3) \mapsto L_a(\psi(r_1, r_2, r_3)),
\end{align*}
and
\begin{align*}
	\tilde{L}_a &\colon \mathrm{L}^{\kern-0.7pt 2} ( [0,+\infty ) \times [0,\pi] \times [0,2\pi ) ) \rightarrow \mathrm{L}^{\kern-0.7pt 2}  ( [0,+\infty ) \times [0,\pi ] \times [0,2\pi ) )\\
	\tilde{L}_a &\colon F(r, \theta, \phi) \rightarrow \tilde{L}_a(F(r, \theta, \phi)).
\end{align*}
If $C$ is the function which transforms co-ordinates, then we have
\begin{align*}
	C &\colon [0,+\infty ) \times [0,\pi] \times [0,2\pi ) \rightarrow \rthree\\
	C &\colon []
\end{align*}


% 	\tilde{L}_a \colon L^\kern-0.7pt2([0,+\infty) \times [0,\pi] \times [0,2\pi)) \rightarrow L^\kern-0.7pt2([0,+\infty) \times [0,\pi] \times [0,2\pi))
Applying chain rule, we get
\begin{align*}
	d
\end{align*}

% 	\begin{align*}
% 		\tilde{L}_1 = -\i\hbar\left(y ( (\partial_z r)\partial_r + (\partial_z \theta)\partial_\theta + (\partial_z \phi)\partial_\phi ) - z((\partial_y r)\partial_r + (\partial_y \theta)\partial_\theta + (\partial_y \phi)\partial_\phi)\right).
% 	\end{align*}
% 	From the expressions for $x$, $y$ and $z$ given before, we see that
% 	\begin{align*}
% 		\partial_z r &= \cos{\theta}, & \partial_y r &= \sin{\phi}\sin{\theta},\\
% 		\partial_z \theta &= -r\sin{\theta}, & \partial_y \theta &= r\sin{\phi}\cos{\theta},\\
% 		\partial_z \phi &= 0,  & \partial_y \phi &= r\cos{\theta}\sin{\theta}.
% 	\end{align*}
% 	Substitution yields
% 	\begin{align*}
% 		\tilde{L_1} &= -\i\hbar\left(r\sin{\theta}\sin{\phi}(\cos{\theta}\partial_r -r\sin{\theta}\partial_\theta) - r\cos{\theta}(\sin{\phi}\sin{\theta}\partial_r + r\sin{\phi}\cos{\theta}\partial_\theta + r\cos{\theta}\sin{\theta}\partial_\phi)\right),\\
% 		& = -\i\hbar\left((-r^2\sin^2{\theta}\sin{\phi} - r^2 \cos^2{\theta}\sin{\phi})\partial_\theta + (-r^2\cos^2{\theta}\sin{\phi})\partial_\phi\right)
% 	\end{align*}
