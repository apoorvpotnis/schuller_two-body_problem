% !TEX root = schuller_two-body_problem.tex
% !TEX program = lualatex
% !BIB program = biber

% In order to compile this file, you need to run the following tools in the following sequence.
% lualatex -> biber -> makeindex -> lualatex -> lualatex
% You can run the following command in a terminal
% lualatex schuller_rigged_hilbert_spaces.tex && biber schuller_rigged_hilbert_spaces && makeindex schuller_rigged_hilbert_spaces && lualatex schuller_rigged_hilbert_spaces.tex && schuller_rigged_hilbert_spaces.tex

% Alternatively, you can run latexmk with the -pdflua option, or use arara as well.

% arara: lualatex: {shell: yes}
% arara: biber
% arara: makeindex
% arara: lualatex: {shell: yes}
% arara: lualatex: {shell: yes}

\documentclass[12pt, a4 paper]{article}
\usepackage{geometry}

\usepackage[ngerman,english]{babel}
\PassOptionsToPackage{math-style=ISO, bold-style=ISO, sans-style=italic, nabla=upright, partial=upright, warnings-off={mathtools-colon,mathtools-overbracket}}{unicode-math}
\usepackage{mathtools, microtype}
\usepackage{setspace}
\usepackage[style=british]{csquotes}

\usepackage{unicode-math}
\setmainfont[%
SizeFeatures={%
    {Size=-8, Font=NewCM08-Book.otf,
        ItalicFont=NewCM08-BookItalic.otf,%
        BoldFont=NewCM10-Bold.otf,%
        BoldItalicFont=NewCM10-BoldItalic.otf,%
        SlantedFont=NewCM08-Book.otf,%
        BoldSlantedFont=NewCM10-Bold.otf,%
        SmallCapsFeatures={Numbers=OldStyle}},
    {Size=8, Font=NewCM08-Book.otf,
        ItalicFont=NewCM08-BookItalic.otf,%
        BoldFont=NewCM10-Bold.otf,%
        BoldItalicFont=NewCM10-BoldItalic.otf,%
        SlantedFont=NewCM08-Book.otf,%
        BoldSlantedFont=NewCM10-Bold.otf,%
        SmallCapsFeatures={Numbers=OldStyle}},
    {Size= 9-, Font = NewCM10-Book.otf,
        ItalicFont=NewCM10-BookItalic.otf,%
        BoldFont=NewCM10-Bold.otf,%
        BoldItalicFont=NewCM10-BoldItalic.otf,%
        SlantedFont=NewCM10-Book.otf,%
        BoldSlantedFont=NewCM10-Bold.otf,%
        SmallCapsFeatures={Numbers=OldStyle}}%
},%
SmallCapsFeatures={Numbers=OldStyle},%
BoldSlantedFont=NewCM10-Bold.otf,%
SlantedFont=NewCM10-Book.otf,%
BoldItalicFont=NewCM10-BoldItalic.otf,%
BoldFont=NewCM10-Bold.otf,%
ItalicFont=NewCM10-BookItalic.otf,%
SlantedFeatures={FakeSlant=0.25},%
BoldSlantedFeatures={FakeSlant=0.25},%
]{NewCM10-Book.otf}
\setsansfont[%
SizeFeatures={%
    {Size= -8, Font=NewCMSans08-Regular.otf,%
        ItalicFont=NewCMSans08-Oblique.otf,%
        BoldFont=NewCMSans10-Bold.otf,%
        BoldItalicFont=NewCMSans10-BoldOblique.otf,%
        SmallCapsFeatures={Numbers=OldStyle},%
    },
    {Size= 8, Font=NewCMSans08-Regular.otf,%
        ItalicFont=NewCMSans08-Oblique.otf,%
        BoldFont=NewCMSans10-Bold.otf,%
        BoldItalicFont=NewCMSans10-BoldOblique.otf,%
        SmallCapsFeatures={Numbers=OldStyle},%
    },
    {Size= 9-, Font=NewCMSans10-Regular.otf,%
        ItalicFont=NewCMSans10-Oblique.otf,%
        BoldFont=NewCMSans10-Bold.otf,%
        BoldItalicFont=NewCMSans10-BoldOblique.otf,%
        SmallCapsFeatures={Numbers=OldStyle},%
    }},
ItalicFont=NewCMSans10-Oblique.otf,%
BoldFont=NewCMSans10-Bold.otf,%
BoldItalicFont=NewCMSans10-BoldOblique.otf,%
SmallCapsFeatures={Numbers=OldStyle},%
SlantedFont=NewCMSans10-Oblique.otf,%
BoldSlantedFont=NewCMSans10-BoldOblique.otf,%
]{NewCMSans10-Regular.otf}
\setmathfont{NewCMMath-Book.otf}[StylisticSet=3, BoldFont = NewCMMath-Bold.otf]
\setmathfont[range={\symfrak,\mathbffrak}, BoldFont = NewCMMath-Bold.otf]{NewCMMath-Regular.otf}
\setmonofont[%
ItalicFont=NewCMMono10-Italic.otf,%
BoldFont=NewCMMono10-Bold.otf,%
BoldItalicFont=NewCMMono10-BoldOblique.otf,%
SlantedFont=NewCMMono10-Regular.otf,%
SlantedFeatures={FakeSlant=0.25},
BoldSlantedFont=NewCMMono10-Bold.otf,%
BoldSlantedFeatures={FakeSlant=0.25},
SmallCapsFeatures={Numbers=OldStyle}]{NewCMMono10-Regular.otf}
\setsansfont[%
SizeFeatures={%
    {Size= -8, Font=NewCMSans08-Regular.otf,%
        ItalicFont=NewCMSans08-Oblique.otf,%
        BoldFont=NewCMSans10-Bold.otf,%
        BoldItalicFont=NewCMSans10-BoldOblique.otf,%
        SmallCapsFeatures={Numbers=OldStyle},%
    },
    {Size= 8, Font=NewCMSans08-Regular.otf,%
        ItalicFont=NewCMSans08-Oblique.otf,%
        BoldFont=NewCMSans10-Bold.otf,%
        BoldItalicFont=NewCMSans10-BoldOblique.otf,%
        SmallCapsFeatures={Numbers=OldStyle},%
    },
    {Size= 9-, Font=NewCMSans10-Regular.otf,%
        ItalicFont=NewCMSans10-Oblique.otf,%
        BoldFont=NewCMSans10-Bold.otf,%
        BoldItalicFont=NewCMSans10-BoldOblique.otf,%
        SmallCapsFeatures={Numbers=OldStyle},%
    }},
ItalicFont=NewCMSans10-Oblique.otf,%
BoldFont=NewCMSans10-Bold.otf,%
BoldItalicFont=NewCMSans10-BoldOblique.otf,%
SmallCapsFeatures={Numbers=OldStyle},%
SlantedFont=NewCMSans10-Oblique.otf,%
BoldSlantedFont=NewCMSans10-BoldOblique.otf,%
]{NewCMSans10-Regular.otf}

\DeclareMathAlphabet{\mathcal}{OMS}{cmsy}{m}{n}
\let\symcal\mathcal

\makeatletter
\g@addto@macro\bfseries{\boldmath}
\makeatother

\usepackage{amsthm}
\theoremstyle{definition}
\newtheorem{thm}{Theorem}
\newtheorem{defn}{Definition}

% \usepackage{breqn}

\title{The Quantum Mechanical Two-body Problem}
% \author{Frederic Schuller\\Apoorv Potnis}
\author{}
\date{\vspace{-5ex}}

\usepackage{silence}
\WarningFilter{latex}{Writing or overwriting file}
\begin{filecontents}[overwrite]{schuller_two-body_problem.bib}

    @unpublished{Schuller,
        author = {Schuller, Frederic and Rea, Simon and Dadhley, Richie},
        institution = {\selectlanguage{ngerman}Friedrich-Alexander-Universität Erlangen-Nürnberg, Institut für Theoretische Physik III\selectlanguage{english}},
        title = {Lectures on Quantum Theory},
        url = {https://docs.wixstatic.com/ugd/6b203f_a94140db21404ae69fd8b367d9fcd360.pdf},
        year = {2019},
        note = {Lecture notes in \texttt{.pdf} format. Lecturer: Prof.\@ Frederic Paul Schuller}
    }

    @misc{SchullerVideos,
        author = {Schuller, Frederic},
        title = {Lectures on Quantum Theory},
        url = {https://youtube.com/playlist?list=PLPH7f_7ZlzxQVx5jRjbfRGEzWY_upS5K6},
        year = {2015},
        note = {Video lectures on YouTube.}
    }

    @misc{Schuller_geometric_notes,
        author = {Schuller, Frederic and Rea, Simon and Dadhley, Richie},
        institution = {\selectlanguage{ngerman}Friedrich-Alexander-Universität Erlangen-Nürnberg, Institut für Theoretische Physik III\selectlanguage{english}},
        title = {Lectures on the Geometric Anatomy of Theoretical Physics},
        url = {https://drive.google.com/file/d/1nchF1fRGSY3R3rP1QmjUg7fe28tAS428/view},
        year = {2017},
        note = {Lecture notes in \texttt{.pdf} format. Lecturer: Prof.\@ Frederic Paul Schuller}
    }

    @misc{Schuller_geometric_videos,
        author = {Schuller, Frederic},
        title = {Lectures on the Geometric Anatomy of Theoretical Physics},
        url = {https://www.youtube.com/playlist?list=PLPH7f_7ZlzxTi6kS4vCmv4ZKm9u8g5yic},
        year = {2016},
        note = {Video lectures on YouTube.}
    }

    @book{Bowers,
        author = {Bowers, Philip},
        isbn = {978-1-108-42976-4},
        publisher = {Cambridge University Press, Cambridge},
        title = {Lectures on Quantum Mechanics},
        year = {2020}
    }

    @book{Reed,
        author = {Reed, Michael and Simon, Barry},
        isbn = {978-0-080-57048-8},
        publisher = {Academic Press, Inc. London},
        title = {Methods of Modern Mathematical Analysis I: Functional Analysis},
        edition = {Revised and Enlarged editon},
        year = {1980},
        volume = 1,
    }

    @misc{MorettiDomain,
        author = {Valter Moretti (\url{https://physics.stackexchange.com/users/35354/valter-moretti})},
        howpublished = {Physics Stack Exchange (\url{https://physics.stackexchange.com})},
        note = {(version: 2014-07-23 00:40:35Z)},
        title = {Answer to the question `What are the proper domains of the position and squared angular momentum operator?'},
        url = {https://physics.stackexchange.com/a/127629/81224}
    }

    @misc{MorettiPeter,
        author = {Valter Moretti (\url{https://physics.stackexchange.com/users/35354/valter-moretti})},
        howpublished = {Physics Stack Exchange (\url{https://physics.stackexchange.com})},
        note = {(version: 2024-05-17 21:03:48Z)},
        title = {Answer to the question `Angular momentum Lie algebra for infinite-dimensional Hilbert spaces'},
        url = {https://physics.stackexchange.com/a/814882/81224}
    }

    @book{MorettiSpectral,
        author = {Moretti, Valter},
        isbn = {978-3-319-70705-1},
        publisher = {Springer Nature Switzerland AG},
        title = {Spectral Theory and Quantum Mechanics: Mathematical Foundations of Quantum Theories, Symmetries and Introduction to the Algebraic Formulation},
        edition = {2},
        year = {2018},
        addendum = {Translated from Italian by Simon G.\ Choissi.}
    }

    @book{MorettiFundamental,
        author = {Moretti, Valter},
        isbn = {978-3-030-18345-5},
        publisher = {Springer Nature Switzerland AG},
        title = {Fundamental Mathematical Structures of Quantum Theory: Spectral Theory, Foundational Issues, Symmetries, Algebraic Formulation},
        year = {20189}
    }

    @book{Teschl,
        author = {Teschl, Gerald},
        isbn = {978-1-4704-1704-8},
        publisher = {American Mathematical Society, Providence, Rhode Island},
        edition = {2},
        title = {Mathematical Methods in Quantum Mechanics: With Applications to Schr\"odinger Operators},
        year = {2014},
        note = {The book is freely available in \texttt{.pdf} format on the author's website at \url{https://www.mat.univie.ac.at/~gerald/ftp/book-schroe/schroe.pdf}.}
    }

    @book{Prugovecki,
        author = {Prugove\v{c}ki , Eduard},
        title = {Quantum Mechanics in Hilbert Space},
        edition = {2},
        year = {1981},
        publisher = {Academic Press, Inc. London},
        isbn = {0-12-566060-X}
    }

    @book{Hall,
        author = {Brian C.\ Hall},
        title = {Quantum Theory for Mathematicians},
        year = {2013},
        publisher = {Springer Science+Business Media New York},
        isbn = {978-1-4614-7115-8}
    }

    @book{Gustafson,
        author = {Gustafson, Stephen J. and Sigal, Israel Michael},
        title = {Mathematical Concepts of Quantum Mechanics},
        edition = {2},
        publisher = {Springer-Verlag Berlin Heidelberg},
        year = {2011},
        isbn = {978-3-642-21865-1}
    }

    @article{Nelson,
        author = {Nelson, Edward},
        doi = {10.2307/1970331},
        journaltitle = {Annals of Mathematics},
        number = {3},
        pages = {572--615},
        publisher = {Annals of Mathematics, Trustees of Princeton University on Behalf of the Annals of Mathematics, Mathematics Department, Princeton University},
        title = {Analytic Vectors},
        volume = {70},
        year = {1959}
    }

\end{filecontents}

\usepackage[sorting=none, hyperref, backref]{biblatex}
\addbibresource{schuller_two-body_problem.bib}
\usepackage{bibentry}
\usepackage{embedall}
\embedfile[desc = bibliography source file]{schuller_two-body_problem.bib}

\usepackage{hyperref}
\hypersetup{linktoc=all, citecolor=red, pdfencoding=auto, psdextra, colorlinks=true, linkcolor=red, breaklinks=true, urlcolor=blue, pdftitle={Schuller – The Quantum Mechanical Two-body Problem}, bookmarksopen=true, pdfauthor={Apoorv Potnis}, pdfsubject={Two-body Problem in Quantum Mechanics}, unicode=true, pdftoolbar=true, pdfmenubar=true, pdfstartview={FitH}, pdfkeywords={Frederic Schuller, Hydrogen Atom, Quantum Mechanics, Mathematical Physics, Lecture Notes, Angular Momentum, Two-body Problem, Kepler problem}}

\usepackage{imakeidx}
\makeindex[intoc]

\usepackage{caption, cleveref, xurl, bookmark}

\newcommand{\ltwo}{\symup{L\kern-0.5pt^2}}
\newcommand{\ltworthree}{\ltwo(\rr^3)}
\newcommand{\ltworsix}{\ltwo(\rr^6)}
\newcommand{\position}{Q}
\newcommand{\momentum}{P}
\newcommand{\rthree}{\symbb{R}^3}
\newcommand{\rr}{\symbb{R}}
\newcommand{\cc}{\symbb{C}}
\newcommand{\nn}{\symbb{N}_0}
\newcommand{\zz}{\symbb{Z}}
\newcommand{\dirac}{\delta}
\newcommand{\hilbert}{\symcal{H}}
\newcommand{\ltwor}{\ltwo(\rr)}
\newcommand{\poly}{\symcal{P}}
\newcommand{\schwartz}{\symcal{S}}
\newcommand{\schwartzr}{\schwartz(\rr)}
\newcommand{\schwartzrthree}{\schwartz(\rr^3)}
\newcommand{\dist}{\schwartz^\times}
\newcommand{\distr}{\dist(\rr)}
\newcommand{\distar}{\schwartz^*(\rr)}
\newcommand{\anti}{\overline{\distr}}
\renewcommand{\i}{\symrm{i}}
\newcommand{\e}{\symrm{e}}
\renewcommand{\pi}{\muppi}
\newcommand{\cinfinity}{\symrm{C}^\infty}
\newcommand{\family}{\symcal{F}}
\newcommand{\lift}{\hat{D}}
\newcommand{\domain}{\symcal{D}}
\newcommand{\identity}{\symrm{id}}
\newcommand{\w}{w\kern-1pt}
% \DeclareMathOperator{\spec}{\sigma}
\newcommand{\spec}{\sigma}
\newcommand{\resolvent}{\symup{\rho}}
\newcommand{\lap}{∆}
\newcommand{\xone}{\symbfit{x}^{(1)}}
\newcommand{\xtwo}{\symbfit{x}^{(2)}}
\newcommand{\rvec}{\symbfit{r}}
\newcommand{\yvec}{\symbfit{y}}
\newcommand{\angsone}{\tilde{L}_1}
\newcommand{\angstwo}{\tilde{L}_2}
\newcommand{\angsthree}{\tilde{L}_3}
\newcommand{\lvecsquare}{\tilde{\symbfit{L}}^2}
\newcommand{\parr}{\frac{\partial}{\partial r}}
\newcommand{\partheta}{\frac{\partial}{\partial \theta}}
\newcommand{\parphi}{\frac{\partial}{\partial \phi}}
\newcommand{\levi}{\epsilon_{ijk}}
\DeclareMathOperator{\spann}{span}
\newcommand{\sutwo}{\operatorname{\symfrak{su}}(2)}
\newcommand{\sltwoc}{\operatorname{\symfrak{sl}}(2;\cc)}
\newcommand{\hrel}{\tilde{H}_\text{rel}}
\newcommand{\lthree}{\tilde{L}_3}
\newcommand{\flme}{F^m_{l,E}}
\newcommand{\SUtwo}{\operatorname{SU}(2)}
\newcommand{\SOthree}{\operatorname{SO}(3)}
\newcommand{\bhilbert}{\hilbert_{\symrm{b}}}
\newcommand{\eigen}{\operatorname{Eig}}
\newcommand{\ltwoa}{\ltwo(\tilde{A})}
\newcommand{\sphe}{Y_l^m}

\DeclarePairedDelimiter{\norm}{\lVert}{\rVert}
\DeclarePairedDelimiter{\abs}{\lvert}{\rvert}

\DeclarePairedDelimiterX\setc[2]{\{}{\}}{\,#1 \;\delimsize\vert\; #2\,}

\newcommand{\der}{\operatorname{d\!}{}}

\newcommand\restr[2]{{% we make the whole thing an ordinary symbol
        \left.\kern-\nulldelimiterspace % automatically resize the bar with \right
        #1 % the function
        \littletaller % pretend it's a little taller at normal size
        \right|_{#2} % this is the delimiter
    }}

\newcommand{\littletaller}{\mathchoice{\vphantom{\big|}}{}{}{}}


% \usepackage{etoolbox}
% \apptocmd{\sloppy}{\hbadness 10000\relax}{}{}
% \emergencystretch=1em

\makeatletter
\g@addto@macro\bfseries{\boldmath}
\makeatother

\usepackage{tikz, tkz-euclide, pgfplots}
\pgfplotsset{compat=1.18}

\usepackage{tikz-cd}
\tikzcdset{arrow style=math font}
\usetikzlibrary{decorations.markings}
\usetikzlibrary {arrows.meta,bending}

\usepackage{tikz-3dplot}

\tdplotsetmaincoords{60}{110}

\pgfmathsetmacro{\rlength}{.8}
\pgfmathsetmacro{\thetavec}{30}
\pgfmathsetmacro{\phivec}{60}

\begin{document}

    \maketitle

    These are lecture notes by Apoorv Potnis of the lecture `\selectlanguage{ngerman}Quantenmechanisches Zweikörperproblem\selectlanguage{english}' or `The Quantum Mechanical Two-body Problem', given by \textbf{Prof.\@ Frederic Paul Schuller}, as the seventeenth lecture in the course `\selectlanguage{ngerman}Theoretische Physik 2: Theoretische Quantenmechanik\selectlanguage{english}' in 2014/15 at the \selectlanguage{ngerman}Friedrich-Alexander-Universität Erlangen-Nürnberg\selectlanguage{english}. While the original lecture is in German, these notes are in English and have been prepared using YouTube's automatic subtitle translation tool. The lecture is available at \url{https://www.youtube.com/watch?v=mcM4S3IMMvI&list=PLPO5pgr_frzTeqa_thbltYjyw8F9ehw7v&index=17} and at \url{https://www.fau.tv/clip/id/4489}.

    The source code, updates and corrections to this document can be found on this GitHub repository: \url{https://github.com/apoorvpotnis/schuller_two-body_problem}. The source code, along with some other files is embedded in this \textsc{pdf}. Comments and corrections can be mailed at \href{mailto:apoorvpotnis@gmail.com}{\texttt{apoorvpotnis@gmail.com}} or opened as an issue in the GitHub repository.

    In these notes, we assume that the reader is already familiar with all of the material covered in Schuller's lectures in his English quantum mechanics series~\cite{SchullerVideos, Schuller}. We shall make frequent use of concepts and results from these lectures without mentioning.

    \tableofcontents

    \section{Introduction}

    \begin{figure}
        \centering
        \begin{tikzpicture}
% 			\node (mone) at (0,0) {\(m_1\)};
            \filldraw [black] (0,0) circle (6pt);
            \node (mone) at (0.4,0.4) {\(m_1\)};
            \filldraw [black] (1.6,-1.6) circle (6pt);
            \node (mtwo) at (1.2,-2) {\(m_2\)};
            \draw [black, dashed, thick] (0,0) -- (1.6,-1.6) node [midway, above right] {\(r\)};
            \tkzDefPoint(0.8,-0.8){O}
            \tkzDrawArc[R, color=black, line width=0.6pt, -{>[bend]}](O,1.1cm)(340,10)
            \tkzDrawArc[R, color=black, line width=0.6pt, -{>[bend]}](O,1.1cm)(160,190)

% 			\draw [black, thick, ->] (0,0) arc [start angle=10, end angle=40, x radius=1.6cm, y radius =1.6cm];

% 			\draw[black, thick, ->, xshift=0.8cm, yshift=-0.8cm] (-30:0.8cm) arc (-30:-10:0.8cm);
        \end{tikzpicture}
        \caption{Classical picture}\label{fig:twobody}
    \end{figure}
    Consider a classical system system consisting of two interacting particles of masses \(m_1\) and \(m_2\), such that the interaction is completely determined by the potential $V(r)$, and the potential depends only on the distance $r$ between the particles. Refer to \cref{fig:twobody}. If we consider the central force problem, like a planet orbiting the sun, then we know we have bound states corresponding to \textit{all} the negative energy values, and scattering states corresponding to all the positive energy values. It turns out that if consider the system quantum-mechanically, then bound states cannot admit all the negative energy solutions. Only certain energy levels are allowed. This can be seen experimentally from the discrete lines in the spectra of atoms, demonstrating the quantised nature of energy levels. In this lecture, we shall consider the quantum-mechanical case of the two-body problem.

    \begin{figure}
        \centering
        \begin{tikzpicture}[line cap=round]
            %		%Grid
            %		\draw[thin, dotted] (0,0) grid (8,8);
            %		\foreach \i in {1,...,8}
            %		{
            %			\node at (\i,-2ex) {\i};
            %		}
            %		\foreach \i in {1,...,8}
            %		{
            %			\node at (-2ex,\i) {\i};
            %		}
            %		\node at (-2ex,-2ex) {0};

% 			% Axis
% 			\draw[thick, -{>[bend]}] (0,0) -- (5,0) node[above] {$r$};
% 			\draw[thick, {<[bend]}-{>[bend]}] (0,-4.5) -- (0,0.5) node[right] {$V(r)$};
% 			\node at (-0.3,0) {$0$};

            \begin{axis}[
                samples=500,
                domain=0.24:4.5,
                xmin=0, xmax=5,
                ymin=-4.4, ymax=0.5,
                axis lines=middle,
                ticks=none,
                width=0.5\textwidth,
% 				height=5cm,
                xlabel={$r$},
                ylabel={$V(r)$},
                x axis line style={thick, -{>[bend]}},
                y axis line style={thick, {<[bend]}-{>[bend]}},
                y label style={anchor=south},
                x label style={anchor=west},
                legend style={draw=none, at={(axis cs:2,-1.8)},anchor=north west},
                legend style={row sep=10pt}
                ]
                \addplot[thick, red] {-exp(-\x)/(\x)};
                \addplot[thick, blue, dashed] {-1/(\x)};
                \addlegendentry{$\displaystyle-\frac{\e^{-r}}{r}$};
                \addlegendentry{$\displaystyle-\frac{1}{r}$};
            \end{axis};


            % Plot Function
    % 		\draw[domain=0.177:7.5, samples=600, variable=\r, very thick] plot ({\r},{0.3/(\r*\r)-0.9/\r});
    % 		\draw[domain=0.6:7.5, samples=500, variable=\r, thick, dashed, red] plot ({\r},{1/(\r*\r)});
% 			\draw[domain=0.24:4.5, samples=500, variable=\r, thick, blue] plot ({\r},{-1/(\r)});
% 			\draw[domain=0.24:4.5, samples=500, variable=\r, thick, red] plot ({\r},{-exp(-\r)/(\r)});

            % Dashed
    % 			\draw[dashed] (2/3,0) -- +(0,-0.65) node[pos=0, above] {$r_o$};
% 			\draw[dashed] (2/3,-0.68) -- (0,-0.68);

            % Nodes
% 			\node at (0.6,3.5) {$V_\text{eff}$};
% 			\node[red] at (1.3,1.5) {$1/r^2$};
% 			\node[blue] at (1.4,-1.5) {$-1/r$};
        \end{tikzpicture}
        \caption{Yukawa and Coulomb potentials}\label{fig:potential}
    \end{figure}
    An example of a potential as described above would be the \textit{Yukawa potential}, defined as
    \[
        V_{\text{Yukawa}}(r) \coloneq a \frac{\exp({-kmr})}{r},
    \]
    where $k$, $m$ and $a$ are constants. $a \in \rr \setminus \{0\}$, $m \geq 0$. Refer to \cref{fig:potential}. According to quantum field theory, very roughly speaking, interaction between particles takes place via a `mediating particle'. If the interaction is mediated by a `scalar field'\footnote{Whatever that means.}, then the mass associated to the particle of that scalar field is the mass $m$ appearing in the Yukawa potential. If we plot a graph of the Yukawa potential for a massive scalar field, then we see that the magnitude of the potential becomes very close to zero after a certain distance. Thus, these interactions are short-ranged. If instead we have $m = 0$, corresponding to a photon, then we get the familiar long-range Coulomb potential
    \[
        V_{\text{Coulomb}}(r) \coloneq a \frac{1}{r}.
    \]

    We also have the finite wall potential $V_{\text{finite wall}}(r) \coloneq a \upTheta(r - r_0)$, where \(\upTheta\) denotes the Heaviside function, and the isotropic harmonic oscillator $V_{\text{ihc}} (r) \coloneq a r^2$.

    \section{Moving to center-of-mass co-ordinates}

    The Hilbert space of the individual particles is \(\ltworthree\) and the Hilbert space of the composite two-body system is \(\ltworthree \otimes \ltworthree\), which is naturally isomorphic to \(\ltworsix\). This can be seen from theorem II.10 of the book of Reed and Simon~\cite[p.~52]{Reed}. If $f, g \in \ltworthree$, then we can define an isomorphism
    \begin{align*}
        I &\colon \ltworthree \otimes \ltworthree \rightarrow \ltworsix,\\
        I &\colon f \otimes g \mapsto fg,
    \end{align*}
    where \(f g\) denotes the point-wise product \[(fg)(\symbfit{x}, \symbfit{y}) \coloneq f(\symbfit{x}) g(\symbfit{y}).\] Tensor products of Hilbert spaces have been discussed in the fourteenth lecture `14: Composite Systems' in the English quantum mechanics series~\cite{SchullerVideos, Schuller}.

    The Hamiltonian for our quantum-mechanical two-body system is
    \[
        H \coloneq -\frac{\hbar^2}{2m_1}\lap_{(1)} - \frac{\hbar^2}{2m_2}\lap_{(2)} + V\left(\norm{\xone - \xtwo}\right),
    \]
    where \(\lap_{(i)}\) is the Laplacian operator which acts on the Hilbert space \(\ltworthree\) of the particle \(i\),\footnote{Actually, \(\lap_{(i)}\) acts on \(\ltworsix\), but we can identify it with \(\ltworthree \otimes \ltworthree\) and \(\lap_{(i)}\) acts non-trivially on one \(\ltworthree\) and trivially on the other.}, and \(\symbfit{x}^{(i)} \in \rthree\) denotes the position vector of the \(i^\text{th}\) particle. Assuming that this Hamiltonian is self-adjoint, we are interested in finding out its spectrum.

    As is the case in classical mechanics, we shall move to the center-of-mass co-ordinates. We can move the center-of-mass of the system in any direction by any value, i.e. the spectrum of the `center-of-mass' position operator is \(\rr\). The discrete energy levels are due to the angular momentum about the center-of-mass. We thus introduce
    \begin{align*}
        \yvec &\coloneq \frac{m_1 \xone + m_2 \xtwo}{m_1 + m_2},\\
        \rvec &\coloneq \xone - \xtwo,\\
        \mu &\coloneq \frac{m_1 m_2}{m_1 + m_2}.
    \end{align*}
    The Hamiltonian then takes the form
    \[
        H \coloneq -\frac{\hbar^2}{2(m_1 + m_2)}\lap_{\yvec} - \frac{\hbar^2}{2\mu}\lap_{\rvec} + V(\norm{\rvec}).
    \]
    The wave function of the system \(\psi\left(\xone, \xtwo\right)\) expressed in the new co-ordinates becomes
    \[
        \phi(\yvec, \rvec) \coloneq \psi\left(\xone(\yvec, \rvec), \xtwo(\yvec,\rvec)\right).
    \]
    The Hamiltonian $H$ can be separated into two parts, $H_\text{free}$ describing the movement of the center-of-mass and $H_\text{rel}$, describing the movement of the particles around the center-of-mass. We have
    \[
        H_\text{free} \coloneq -\frac{\hbar^2}{2(m_1 + m_2)}\lap_{\yvec} \qquad \text{ and } \qquad H_\text{rel} \coloneq - \frac{\hbar^2}{2\mu}\lap_{\rvec} + V(\norm{\rvec}).
    \]
    We know that $\spec{(H_\text{free})} = \rr_{\geq 0}$, so the task that remains is to find $\spec{(H_\text{rel})}$, provided that $H_\text{rel}$ is self-adjoint, depending on $V(\norm{\rvec})$. We see that $H_\text{rel}$ is spherically symmetric. Thus, we shall shortly move on to spherical co-ordinates to separate the angular and radial dependencies. This shall reduce our three-dimensional problem to a one-dimensional one.

    \section{Spectrum of angular momentum operators}

    We know that the orbital angular momentum operators are self-adjoint and defined on their Stone domains. They are defined as the generators of a rotation about an axis. We shall instead consider the restrictions of the these operators on a common core, the Schwartz space \(\schwartzrthree\), to avoid running into technical issues. \(\schwartzrthree\) is dense in the Stone domains and $\ltworthree$. These operators are essentially self-adjoint on the Schwartz space. This machinery has been explained in the twelfth lecture `12: Stone's Theorem and Construction of Observables' in the English quantum mechanics series~\cite{SchullerVideos, Schuller}.

    Let $i \in \{1,2,3\}$. We thus have the Cartesian angular momentum operators
    \begin{align*}
        L_i &\colon \schwartzrthree \rightarrow \schwartzrthree,\\
        L_i &\colon \psi \mapsto \levi\position_j\momentum_k\psi,
    \end{align*}
    where $\position_j$ and $\momentum_k$ denote the $j^\text{th}$ and $k^\text{th}$ position and momentum operators respectively. The indices run over 1, 2 and 3.
    We thus have
    \begin{align*}
        L_1 &= -\i\hbar(y\partial_z - z\partial_y),\\
        L_2 &= -\i\hbar(z\partial_x - x\partial_z),\\
        L_3 &= -\i\hbar(z\partial_y - y\partial_x).
    \end{align*}

    Consider a Lie algebra over \(\cc\) with the basis \(\{J_1, J_2, J_3\}\), where \(J_i\)'s are linear operators on a \textit{finite-dimensional} Hilbert space and the Lie bracket is given by the commutator. Furthermore, let \([J_i, J_j] = \i \epsilon_{ijk} J_k\) for all \(i, j, k \in \{1, 2, 3\}\). In the thirteenth lecture `13: Spin' of the English quantum mechanics series~\cite{Schuller, SchullerVideos}, we analysed the spectrum of such a system. But it should be noted that without any further justifications, such an analysis \textit{cannot} be done when the vector spaces in question are infinite-dimensional. The Schwartz space, which we need to consider for a spectral analysis of the orbital angular momentum operators, is infinite-dimensional. The justification we need is something Schuller briefly alludes to in the end of the twelfth lecture, but does not elaborate on it any further in the lecture series. We need to consider a number of finite-dimensional subspaces of the Schwartz space, each of which is invariant under the action of the group corresponding to rotations along an axis.\footnote{Let \(D_j(\alpha) \colon \rthree \rightarrow \rthree\) be the operator which describes a counter-clockwise rotation about the \(j\)\textsuperscript{th} axis by an angle \(\alpha\). The corresponding group here then is \(\{D_j(\alpha) \mid \alpha \in [0, 2\pi)\}\) with the group operation being executing the rotations about the axis one after other.} Our treatment here is based on an answer of Valter Moretti~\cite{MorettiPeter}.

    The following discussion for this section requires some rudimentary definitions from topology and representation theory; more precisely, one needs to know the following notions and concepts. Topological: Hausdorffness, compactness and simple connectedness. Representation theory: Irreducible unitary representations, the Lie groups \(\SUtwo\), \(\SOthree\) and \(\operatorname{SL}(2;\cc)\), and the Lie algebras \(\sutwo\) and \(\sltwoc\). In my (Apoorv) opinion, understanding these notions from Wikipedia is sufficient for the present purposes. One can also take a look at `The Geometrical Anatomy of Physics' lectures of Schuller~\cite{Schuller_geometric_videos, Schuller_geometric_notes} for a more detailed view of some of these concepts.

    \subsection{Justification for considering simultaneous eigenvectors}

    Consider a Lie algebra over \(\cc\) with the basis \(\{J_1, J_2, J_3\}\), where each \(J_i\) is a linear operator on a Hilbert space \(\hilbert\), for \(i \in \{1, 2, 3\}\), and the Lie bracket is given by the commutator. Furthermore, we assume that the following conditions are satisfied.
    \begin{enumerate}
        \item Each \(J_i\) is essentially self-adjoint on a common core.
        \item \(\symbfit{J}^2\) is also essentially self-adjoint on the same core, with \(\symbfit{J}^2 \coloneq \sum_{i=1}^3 J_i \circ J_i\).
        \item \([J_i, J_j] = \i \epsilon_{ijk} J_k\) for all \(i, j, k \in \{1, 2, 3\}\).
    \end{enumerate}
    We denote the Lie algebra by \(\symcal{A} = \spann_{\cc}\{J_1, J_2, J_3\}\). We need to show that each \(J_i\) has a point spectrum.\footnote{Recall that the point spectrum was defined as the set of all the eigenvalues of the operator. In this case, the spectrum shall turn out to be pure point. In Schuller's definition of the pure point spectrum, the elements, i.e.\ the eigenvalues, can have countably infinite multiplicities. In the case of angular momentum, the eigenvalues of the operators \(L_3\) and \(\symbfit{L}^2\) shall be revealed to have infinite multiplicities. It seems to me (Apoorv) that the definitions of different types of spectra have slight variations in the literature. The reader is requested to check the definition a particular source is using while consulting the literature.}

    We now state a theorem of Edward Nelson, as stated in a book of Moretti~\cite[p.~291]{MorettiFundamental}, with slight modifications.
    \begin{thm}[Nelson]
        Consider a real \(n\)-dimensional Lie algebra \(V\) of operators \(-\i S_1\), \ldots, \(-\i S_n\), where each \(S_i\) with \(i \in \{1, \ldots, n\}\) is essentially self-adjoint on the Hilbert space \(\hilbert\), defined on a common invariant and dense subspace \(\domain \subset \hilbert\), with the usual commutator as the Lie bracket.
        Let \(-\i S_1, \ldots, -\i S_n \in V\) be a basis of \(V\) and define Nelson's operator with domain \(\domain\) as
        \[
            \Delta_{\symrm{N}} \coloneq \sum_{i=1}^{n} S_i^2.
        \]
        If \(\Delta_{\symrm{N}}\) is essentially self-adjoint, then there exists a strongly-continuous unitary representation
        \begin{align*}
            U &\colon G_V \rightarrow V,\\
            U &\colon g \mapsto U_g
        \end{align*}
        on \(\hilbert\) of the unique simply-connected Lie group \(G_V\) with the Lie algebra \(V\).
        \(U\) is uniquely determined by the fact that the closures \(\overline{S}\), for every \(-\i S \in V\), are the self-adjoint generators of the representations of the one-parameter subgroups of \(G_V\).
    \end{thm}
    A proof of the above theorem can be found in the original paper of Nelson~\cite{Nelson}. In our case, we see that the conditions of Nelson's theorem are satisfied, with the real Lie algebra \(V = \spann_{\rr}\{-\i J_1, -\i J_2, -\i J_3\}\), with the commutator relation \[[-\i J_i, -\i J_j] = \epsilon_{ijk}(-\i J_k),\] the Hilbert space being \(\ltworthree\), and the common domain or core  being the Schwartz space \(\schwartzrthree\). \(V\) is isomorphic to \[\displaystyle\sutwo = \spann_{\rr}\left\{-\frac{\i\sigma_1}{2}, -\frac{\i\sigma_2}{2}, -\frac{\i\sigma_3}{2}\right\}\] as a Lie algebra, where \(\sigma_i\) denote the Pauli matrices, with the commutator relation \[\displaystyle\left[-\frac{\i\sigma_i}{2}, -\frac{\i\sigma_j}{2}\right] = \epsilon_{ijk}\left(-\frac{\i\sigma_k}{2}\right).\] The isomorphism maps \(-\i J_i\) to \(-\i \sigma_i/2\). \(\sutwo\) is the Lie algebra of both \(\SUtwo\) and \(\SOthree\). There is a two-to-one correspondence from \(\SUtwo\) to \(\SOthree\). \(\SOthree\) is not simply connected. The unique simply-connected Lie group associated to \(\sutwo\) is \(\SUtwo\). Thus, the closures of \(J_i\)'s are the self-adjoint generators of the representations of the one-parameter subgroups of \(\SUtwo\). These define a unique strongly continuous representation of \(\SUtwo\) in \(\ltwor\).\footnote{Side note: Since \(\symfrak{s\kern-0.6ptu}(2) + \i\,\symfrak{s\kern-0.6ptu}(2) = \symfrak{s\kern-0.8ptl}(2;\cc)\), we have that \(\symcal{A} = V + \i V\). \(\symcal{A}\) is isomorphic to \(\symfrak{s\kern-0.8ptl}(2;\cc)\) as a Lie algebra.} Note that this representation defines bounded linear functions on \(\ltwor\).

    We now state a version of the Peter--Weyl theorem, as stated in a book of Moretti~\cite[p.~287]{MorettiFundamental}, with slight modifications.

    \begin{thm}[Peter--Weyl]
        Let \(G\) be a compact Hausdorff group, a compact Lie group in particular, and let
        \begin{align*}
            U &\colon G \rightarrow B(\hilbert),\\
            U &\colon g \mapsto U_g,
        \end{align*}
        be a strongly-continuous unitary representation of $G$ on the Hilbert space $\hilbert \neq 0$, with $B(\hilbert)$ denoting the set of all bounded linear functions on $\hilbert$. The following statements then hold.
        \begin{enumerate}
            \item If \(U\) is irreducible, then \(\hilbert\) is finite.
            \item If \(U\) is not irreducible, then the orthogonal Hilbert decomposition
            \[
                \hilbert = \bigoplus_{n\in \symcal{N}} \hilbert_n
            \]
            holds, where \(\hilbert_n\) are pairwise orthogonal non-trivial closed subspaces of finite dimension, all invariant under \(U\), and \(\symcal{N}\) is a possibly infinite subset of the natural numbers.
            \item Every map \(\restr{U}{\hilbert_n} \colon \hilbert_n \rightarrow \hilbert_n\) is an irreducible representation of \(G\).
        \end{enumerate}
    \end{thm}

    \(\operatorname{SU(2)}\) is a compact group. And thus, our strongly-continuous unitary representation of \(\SUtwo\) on \(\ltworthree\) decomposes \(\ltworthree\) into finite dimensional subspaces. Let \(\ltworthree_n\) denote such a subspace. Since \(\ltworthree\) is an infinite dimensional Hilbert space, the number of such finite dimensional subspaces must be countably infinite. Thus,
    \[
        \ltworthree = \bigoplus_{n \in \nn} \ltworthree_n.
    \]
    Define \(\schwartzrthree_n \coloneq \schwartzrthree \cap \ltworthree_n\), for all \(n \in \nn\). Thus,
    \[
    \schwartzrthree = \bigoplus_{n \in \nn} \schwartzrthree_n.
    \]
    Each \(\schwartzrthree_n\) is invariant under the action of \(J_i\)'s. This was the procedure Schuller briefly mentioned. Now, since \[\restr{J_i}{\schwartzrthree_n} \colon \schwartzrthree_n \rightarrow \schwartzrthree_n\] acts on finite dimensional spaces, it spectrum consists of only eigenvalues, i.e.\ each \(\restr{J_i}{\schwartzrthree_n}\) has a pure point spectrum. The spectrum of \(J_i\)'s just the union of these individual eigenvalues, i.e.\
    \[
        \spec(J_i) = \bigcup_{n \in \nn} \spec\left(\restr{J_i}{\schwartzrthree_n}\right).
    \]
    Thus, the usual spectral analysis for angular momentum operators using ladder operators holds, since all \(\schwartzrthree_n\) are finite-dimensional. Thus, using the theorems of Nelson and Peter--Weyl, and the subsequent ladder operator analysis, we get that $J_i$'s have a pure point spectrum, and that the common eigenvectors of $\symbfit{J}^2$ and $J_3$ come as families $\psi_{j(j+1), m}$, where $m = -j, -j+1, \ldots, j-1, +j$. The eigenvalue $j(j+1)$ is associated to $\symbfit{J}^2$ and $m$ is associated to $J_3$ with $j \in \symbb{N}_0/2$.

    Actually, as we only discuss the orbital angular momentum operators, i.e.\ the \(L_i\)'s in this lecture, we could have used a slightly different approach to justify the procedure using ladder operators. We know that the spectrum of each \(L_i\) is entirely pure point and contained within the integers, i.e.\ for all \(i \in \{1, 2, 3\}\),
    \[
        \spec(L_i) \subseteq \zz.
    \]
    We now need a version of the spectral mapping theorem, which we state here, referring to theorem~3.17 of the book of Teschl~\cite[p.~118]{Teschl}.
    \begin{thm}[Spectral Mapping]
        Let \(\Xi\) be a self-adjoint operator on a Hilbert space \(\hilbert\), and let \(f \colon \rr \rightarrow \cc\) be a continuous function. Then
        \[
            \spec((f(\Xi)) = f(\spec(\Xi)),
        \]
        if \(\abs{f(\lambda)} \rightarrow \infty\) as \(\abs{\lambda} \rightarrow \infty\).
    \end{thm}
    Here, \(f(\Xi)\) is defined using the spectral calculus. By applying the spectral mapping theorem, we get that complex polynomials in \(L_i\) have the same spectrum as \(L_i\), i.e.\ if \(p\) is a complex polynomial, then
    \[
        \spec(p(L_i)) = p(\spec(L_i)).
    \]
    If we take \(p(x) = x^2\), then we get that \[\spec(L_i^2) \subseteq \zz,\] and \[\spec(L_1^2 + L_2^2 + L_3^2) \subseteq \zz,\] as \(\spec(L_i) \subseteq \zz\). We now need the following result which has been asked as problem~4.5 in the book of Teschl~\cite[p.~138]{Teschl}.
    \begin{thm}
        If two commuting self-adjoint operators on a complex Hilbert space have a pure point spectrum, then there exists a common orthonormal eigenbasis of both the operators.
    \end{thm}
    This justifies the assumption of considering simultaneous eigenvectors of \(\symbfit{L}^2 \coloneq L_1^2 + L_2^2 + L_3^2\) and \(L_3\) in the usual procedure using ladder operators. Note that this \textit{does not} mean that \(l\) and \(m\) specify a unique \(\psi_{l(l+1), m}\). The eigenvalues \(l\) and \(m\) are degenerate countably infinitely, and a choice of orthonormal vectors in each eigenspace must be made to specify an eigenbasis for the whole Hilbert space.

    \subsection{Moving to spherical co-ordinates}

    \begin{figure}
        \centering
        \begin{tikzpicture}[scale=4,tdplot_main_coords]
            \coordinate (O) at (0,0,0);
            \draw[thick,->] (0,0,0) -- (0.7,0,0) node[anchor=north east]{$x$};
            \draw[thick,->] (0,0,0) -- (0,0.7,0) node[anchor=north west]{$y$};
            \draw[thick,->] (0,0,0) -- (0,0,0.7) node[anchor=south]{$z$};
            \tdplotsetcoord{P}{\rlength}{\thetavec}{\phivec}

            \draw[dashed, thick] (O) -- (Pxy);
            \draw[dashed, thick] (P) -- (Pxy);
            \tdplotdrawarc[thick]{(O)}{0.2}{0}{\phivec}{anchor=north}{$\phi$}
            \tdplotsetthetaplanecoords{\phivec}
            \tdplotdrawarc[thick, tdplot_rotated_coords]{(0,0,0)}{0.5}{0}%
            {\thetavec}{anchor=south west}{$\theta$}
            \draw[->,color=red, thick] (O) -- (P) node[above right] {$\rvec$};
            \node (r) at (0,0.15,0.2) {\(r\)};
        \end{tikzpicture}
        \caption{Spherical co-ordinates}\label{fig:sphericalcoordinates}
    \end{figure}

    We have the spherical co-ordinate system as shown in \cref{fig:sphericalcoordinates}. Let
    \[
        A \coloneq [0, +\infty) \times [0,\pi] \times [0,2\pi).
    \]
    Let $T$ be defined as
    \begin{align*}
        T &\colon A \rightarrow \rthree,\\
        T &\colon (r, \theta, \phi) \mapsto (x, y, z),
    \end{align*}
    where
    \begin{align*}
        x &= r \cos{\phi} \sin{\theta},\\
        y &= r \sin{\phi} \sin{\theta},\\
        z &= r \cos{\theta}.
    \end{align*}
    Thus,
    \[
        T(r, \theta, \phi) \coloneq (r \cos{\phi} \sin{\theta}, \sin{\phi} \sin{\theta}, \cos{\theta}).
    \]
    If $F \in \schwartz(A)$ corresponds to $\psi$ in spherical co-ordinates\footnote{\,Note that even though $T$ is only surjective and not injective, $T^{-1}(\psi)$ is a measure zero set for any $\psi \in \schwartzrthree$. Thus, we don't need to make any further identifications in $\schwartz(A)$.}, we have
    \[
        F = \psi \circ T,
    \]
    and the chain rule gives us
    \[
        F'(r, \theta, \phi) = \psi'(x,y,z) \cdot T'(r, \theta, \phi),
    \]
    where $(x,y,z) = T(r, \theta, \phi)$. We thus have
    \[
        \partial_\phi = (\partial_\phi x)\partial_x + (\partial_\phi y)\partial_y + (\partial_\phi z)\partial_z.
    \]
    Substituting $x = r \cos{\phi} \sin{\theta}$, $y = r \sin{\phi} \sin{\theta}$ and $z = r \cos{\theta}$, we get
    \begin{align*}
        \partial_\phi &= -(r \sin{\theta} \sin{\phi})\partial_x + (r \sin{\theta} \cos{\phi})\partial_y\\
        \partial_\phi &= x\partial_2 - y\partial_1.
    \end{align*}
    Thus, as expected on physical grounds, we get
    \[
        \tilde{L}_3 = -\i\hbar\partial_\phi,
    \]
    where $\tilde{L}_3 \colon \schwartz(A) \rightarrow \schwartz(A)$ denotes the angular momentum operator in spherical co-ordinates.
    We now restrict the domain of $T$ suitably such that it becomes injective and $T^{-1}$ becomes a well defined function. More precisely, we take the domain to be $\tilde{A} \coloneq (0, +\infty) \times (0,\pi) \times (0,2\pi)$. This does not cause any problems as we are excluding only measure zero sets which are irrelevant when considering wave-functions. If we define $\tilde{\rr}^3$ as $\rthree$ excluding the axes, we have $T^{-1} \colon \tilde{\rr}^3 \rightarrow \tilde{A}$, $T^{-1} \colon (x, y, z) \mapsto (r, \theta, \phi)$ as
    \begin{align*}
        r      &= \sqrt{x^2 + y^2 + z^2},\\
        \theta &= \cos^{-1}{\left(\frac{z}{\sqrt{x^2 + y^2 + z^2}}\right)},\\
        \phi   &= \cos^{-1}{\left(\frac{x}{\sqrt{x^2 + y^2}}\right)}.
    \end{align*}
    Applying chain rule to $\psi = F \circ T^{-1}$ gives us
    \[
        \partial_i = (\partial_i r)\partial_r + (\partial_i \theta)\partial_\theta + (\partial_i \phi)\partial_\phi.
    \]
    Calculating the partial derivatives and substituting the above in $L_1 = -\i\hbar(y\partial_z - z\partial_y)$ and $L_2 = -\i\hbar(z\partial_x - x\partial_z)$ finally gives us
    \begin{align*}
        \angsone &\coloneq \i\hbar\left(\cos{\phi}\cot{\theta}\frac{\partial}{\partial\phi} + \sin{\phi}\frac{\partial}{\partial\theta}\right),\\
        \angstwo &\coloneq \i\hbar\left(-\sin{\phi}\cot{\theta}\frac{\partial}{\partial\phi} + \cos{\phi}\frac{\partial}{\partial\theta}\right),\\
        \angsthree &\coloneq -\i\hbar \frac{\partial}{\partial\phi}.
    \end{align*}
    We also have
    \[
        \lvecsquare F = -\frac{1}{\sin{\theta}}\frac{\partial}{\partial \theta}\left(\sin{\theta}\frac{\partial}{\partial\theta}F\right) -\frac{1}{\sin{\theta}}\frac{\partial}{\partial\phi}\left(\frac{1}{\sin{\theta}}\frac{\partial}{\partial\theta}F\right),
    \]
    and
    \[
        \tilde{H}_\text{rel} F = -\frac{\hbar^2}{2\mu}\left(\frac{1}{r}\left(\frac{\partial}{\partial r}\right)^2 (rF) - \frac{\lvecsquare}{r^2}F\right) + V(r)F.
    \]

    We now observe that $\lvecsquare F$ is dependent only on the angular part of $F$. So, if we multiply $F$ by a function $f(r)$ of $r$ alone, without any dependencies on $\theta$ and $\phi$, then we have that $\lvecsquare(f\cdot F) = f \cdot \lvecsquare F$. Because of this observation, we attempt to decompose the the Hilbert space into radial and angular parts.

    We know that \[\rthree = \symbb{S}^2 \times [0, +\infty)\] as sets. The standard Lebesgue measure on $\rthree$ can be seen as the tensor product of the measures on $\symbb{S}^2$ and $[0, \infty)$. Let $\symcal{B}(\symbb{S}^2)$ be the Borel $\sigma$-algebra on $\symbb{S}^2$, generated by the standard induced subset topology on $\symbb{S}^2$, and let $\symcal{B}([0,+\infty))$ be the Borel $\sigma$-algebra on $[0, +\infty)$ generated by the standard induced subset topology on $[0, +\infty)$. We know that $\symbb{S}^2 = [0, \pi] \times [0, 2\pi),$ as sets. Every element of $\symcal{B}(\symbb{S}^2)$, i.e.\@ every measurable set of $\symbb{S}^2$, can be expressed as a countable union of sets $\Theta \times \Phi$, where $\Theta \coloneq (\theta_1, \theta_2)$ and $\Phi \coloneq (\phi_1, \phi_2)$ are open sets in $[0, \pi]$ and $[0, 2\pi)$. Let
    \[
        \nu \colon \symcal{B}(\symbb{S}^2) \rightarrow \rr
    \]
    be the measure defined on $\symbb{S}^2$ by
    \[
        \nu((\Theta, \Phi)) \coloneq (\phi_2 - \phi_1) \int_{\theta_1}^{\theta_2}\sin{\theta}\der \theta.
    \]
    $\nu$ is known as the standard spherical measure on $\symbb{S}^2$. It can be easily seen that $\nu((\Theta, \Phi))$ is nothing but the area of a surface element on the sphere which subtends the angles $\Theta$ and $\Phi$ at the origin in the polar and azimuthal directions respectively. We define the measure
    \[
        \xi \colon \symcal{B}([0, +\infty)) \rightarrow \rr
    \]
    on $[0, +\infty)$ by
    \[
        \xi((a, b)) \coloneq \int_a^b r^2\! \der r,
    \]
    with $a, b \in (0, +\infty)$. Note that in both the cases, we have defined the measure on open sets. Every element of the Borel $\sigma$-algebra can be expressed as a countable union or a countable intersection of the open sets. The standard Lebesgue measure $\mu$ on $\rthree$ can be expressed as
    \[
        \mu = \nu \otimes \xi.
    \]
    Applying theorem II.10 of the book of Reed and Simon~\cite[p.~52]{Reed} again, we see that
    \begin{align}
        \ltwo(\rthree, \mu) = \ltwo(\symbb{S}^2, \nu) \otimes \ltwo([0, +\infty), \xi).
    \end{align}
    We define the operators \[\lvecsquare_{\,\symbb{S}^2} \colon \symrm{C}^\infty(\symbb{S}^2, \nu) \rightarrow \symrm{C}^\infty(\symbb{S}^2, \nu)\] and \[\tilde{L}_{3, \symbb{S}^2} \colon \symrm{C}^\infty(\symbb{S}^2, \nu) \rightarrow \symrm{C}^\infty(\symbb{S}^2, \nu).\] These are simply the angular parts of the $\lvecsquare$ and $\tilde{L}_3$ operators, which anyway act on $\symrm{C}^\infty(\symbb{S}^2, \nu)$ since they did not have any radial dependence.
    \begin{align*}
        \lvecsquare_{\,\symbb{S}^2} &= -\i\hbar\partial_\phi,\\
        \tilde{L}_{3, \symbb{S}^2}  &= -\frac{1}{\sin{\theta}}\frac{\partial}{\partial \theta}\left(\sin{\theta}\frac{\partial}{\partial\theta}\right) -\frac{1}{\sin{\theta}}\frac{\partial}{\partial\phi}\left(\frac{1}{\sin{\theta}}\frac{\partial}{\partial\theta}\right).
    \end{align*}
    We use the fact that $\symrm{C}^\infty(\symbb{S}^2, \nu)$ is dense in $\ltwo(\symbb{S}^2, \nu)$. It can be seen that these operators are symmetric.

    We need the fact that $\lvecsquare$ is essentially self-adjoint on $\symrm{C}^\infty(\symbb{S}^2)$ \cite{MorettiDomain}. In order to fully determine $\spec(H_\text{rel})$, along with the continuous spectrum, i.e.\@ the part containing the generalised eigenvalues, we would need to lift the eigenvalue equation to the distribution space and solve it.\footnote{\,This can be in principle be achieved using Gelfand's rigged Hilbert space formalism, which was introduced in an earlier lecture. The lecture itself can be found on this link: \url{https://www.youtube.com/watch?v=FNJOyxOp3Ik&list=PLPO5pgr_frzTeqa_thbltYjyw8F9ehw7v&index=8}. The notes for this lecture can be found here: \url{https://github.com/apoorvpotnis/schuller_rigged_hilbert_spaces/blob/main/schuller_rigged_hilbert_spaces.pdf}.} However, we already know from experiments that the hydrogen atom has discrete lines in its observed spectrum. Thus, we shall first try to find the point spectrum
    \[
        \spec_{\text{p}}(H_\text{rel}) = \setc*{E \in \rr}{H_\text{rel} F = E F},
    \]
    assuming it exists, which was historically a major validation for the quantum theory.

    \subsection{Spherical harmonics}

    \(\schwartzrthree\), \(\schwartz(\tilde{\rr}^3)\), \(\schwartz(A)\), and \(\schwartz(\tilde{A})\) are all isomorphic to each other. Thus, the eigenvalues and eigenvectors of \(\symbfit{L}^2\) and \(L_i\)'s, and \(\lvecsquare\) and \(\tilde{L}_i\)'s are the same, respectively. The self-adjointness is unaffected by these co-ordinate changes as well. Thus, we know that the common eigenvectors of $\lvecsquare$ and $\tilde{L}_3$ come as families $\psi\kern0.5pt_l^m \in \schwartz(\tilde{A})$, where $m = -l, -l+1, \ldots, l-1, l$ and $l \in \nn$. We have
    \begin{align*}
        \lvecsquare \psi\kern0.5pt_l^m &= l(l+1)\psi\kern0.5pt_l^m,\\
        \tilde{L}_3 \psi\kern0.5pt_l^m &= m \psi\kern0.5pt_l^m.
    \end{align*}
    Note that $l$ can take only integer values, not half-integer values as we are dealing with the orbital angular momentum operator. In spherical co-ordinates, it turns out that the solutions $\psi\kern0.5pt_l^m$ are given by
    \[
        \psi\kern0.5pt_l^m (\tilde{\symbfit{x}} (r, \theta, \phi)) = k \cdot Y^m_l (\theta, \phi) \cdot f(r),
    \]
    where
    \[
        Y^m_l (\theta, \phi) \coloneq \frac{(-1)^m}{2^l l!}\left(\frac{(2l+1) (l-m)!}{4\pi (l+m)!}\right)^{\frac{1}{2}}\!\cdot\e^{\i m\phi}\cdot\!(\sin{\theta})^m \left(\frac{\partial}{\partial \cos{\theta}}\right)^{l+m}\!\!\!(\cos^2{\theta} -1)^l
    \]
    and $\tilde{\symbfit{x}} = T^{-1}(\symbfit{x})$. These $\sphe$'s are called as \textit{spherical harmonic functions}. \(\sphe \in \ltwoa\).\footnote{We have identified \(\ltworthree\) and \(\ltwoa\).} Since the operators $\lvecsquare$ and $\tilde{L}_3$ act only on the angular components and not the radial components when expressed in spherical co-ordinates, we have that
    \begin{align*}
        \lvecsquare \sphe &= l(l+1) \sphe,\\
        \tilde{L}_3 \sphe &= m \sphe,
    \end{align*}
    i.e.\@ the spherical harmonic functions are eigenvectors of $\lvecsquare$ and $\tilde{L}_3$. These functions actually form an orthonormal basis of the Hilbert space $\symrm{L}\kern-0.7pt^2(\symbb{S}^2)$ consisting of complex square-integrable functions defined on the 2-sphere.
    \[
        \langle \sphe , Y_{l'}^{m'} \rangle = \delta\kern0.5pt_{l, l'}\cdot\delta\kern0.4pt_{m, m'}.
    \]
    We request that the reader consult section 14.2 of the book of Bowers for a derivation of these spherical harmonic functions~\cite[p.~200]{Bowers}.

    \section{Spectrum of the Hamiltonian}

    For a wide range of potentials \(V(\norm{\rvec})\), the Hamiltonian \(\hrel\) turns out to be self-adjoint. In the case we are most interested in, the hydrogen atom, the potential is the familiar Coulomb potential. It turns out that \(\hrel\) is self-adjoint in that case. So, for the further analysis, we assume that \(\hrel\) is self-adjoint on \(\ltwoa\). Depending on the spectrum of \(\hrel\), common orthonormal eigenvectors of \(\lvecsquare\), \(\tilde{L}_i\), and \(\hrel\) can possibly exist. Note that even though \(\lvecsquare\) and \(\tilde{L}_i\) have pure point spectra, \(\hrel\) may have continuous spectra. As discussed previously, the classic application of the quantum two-body problem is the hydrogen atom, for which we know \textit{experimentally} that some of its spectrum is point. Thus, assuming \textit{a priori} that some of the spectrum of \(\hrel\) is point, we try to find it and the eigenvectors corresponding to these eigenvalues.

    Let \(\domain_P\) be the domain of an operator \(P \colon \domain_P \rightarrow \hilbert\), where \(\hilbert\) is a Hilbert space. We define the eigenspace \(\eigen_P(\lambda)\) of an eigenvalue \(\lambda \in \spec_{\symrm{p}}(P)\) to be the eigenvectors corresponding to the eigenvalue \(\lambda\).\footnote{Note again that the eigenvalues can have infinite multiplicities.} Thus,
    \[
        \eigen_P(\lambda) \coloneq \setc*{\psi \in \domain_P}{\psi \in \hilbert, P\psi = \lambda\psi}.
    \]
    We thus define
    \[
        %\eigen_P(\spec_{\symrm{p}}) \coloneq \overline{\spann{\setc*{ \sum_{k \in \symcal{K}} a_k v_k}{a_k \in \cc, v_k \in \bigcup_{\lambda \in \spec_{\symrm{p}}(P)} \eigen_P(\lambda), \symcal{K} \subset \nn }}}.
        \eigen(P) \coloneq \bigcup_{\lambda \in \spec_{\symrm{p}}(P)} \eigen_P(\lambda).
    \]

    Let us restrict the Hilbert space \(\ltworthree\) to the Hilbert \(\bhilbert\) space formed by the common eigenvectors of \(\lvecsquare\), \(\tilde{L}_i\), and \(\hrel\).\footnote{We choose ``b'' as a subscript in the label \(\bhilbert\) as this space shall later turn out to the space of \textit{bound states}.} Thus,
    \[
        \bhilbert \coloneq \eigen(\lvecsquare) \cap \eigen(\tilde{L}_i) \cap \eigen(\hrel).
    \]
    In order to find the elements of \(\bhilbert\), we consider the following equations.
    \begin{align*}
        \hrel \flme &= E \flme,\\
        \lvecsquare \flme &= l(l+1) \flme,\\
        \lthree \flme &= m \flme.
    \end{align*}
    Here, \(\flme \in \ltwoa\) is a simultaneous eigenvector of \(\hrel\), \(\lvecsquare\) and \(\lthree\), which we assume to exist, indexed by \(l\), \(m\) and \(E\); \(\flme \in \bhilbert\). We already know that the simultaneous eigenvectors of \(\lvecsquare\) and \(\lthree\) are spherical harmonics, and they don't have any radial dependency---they depend only on the polar and the azimuthal angles. Thus, we can write an \textit{ansatz} for \(\flme\) by multiplying the spherical harmonics by a radial function. Thus, we consider the following \textit{ansatz}.
    \[
        \flme(r, \theta, \phi) = \frac{u_E(r)}{r}\sphe(\theta, \phi).
    \]
    We have multiplied by a factor of \(1/r\), as this shall prove to be useful shortly. Recall that \(\hrel\) had the following expression.
    \[
        \tilde{H}_\text{rel} F = -\frac{\hbar^2}{2\mu}\left(\frac{1}{r}\left(\frac{\partial}{\partial r}\right)^2 (rF) - \frac{\lvecsquare}{r^2}F\right) + V(r)F
    \]
    for an \(F \in \ltwoa\). Thus, using this expression, and the eigenvalue equations, we get that
    \[
        -\frac{\hbar^2}{2\mu} \frac{1}{r} u_E''(r) \sphe + \frac{\hbar^2}{2\mu} \frac{l(l+1)}{r^2} \frac{u_E}{r} \sphe + V(r) \frac{u_E}{r} \sphe = E \frac{u_E}{r} \sphe.
    \]
    Simplifying, we get
    \[
        -\frac{\hbar^2}{2\mu} \lap u_E(r) + \left(\frac{\hbar^2}{2\mu} \frac{l(l+1)}{r^2} + V(r)\right) u_E = E u_E.\tag{\textasteriskcentered}\label{eigen}
    \]
    Upon observing the above equation, we notice that the first term can be interpreted as corresponding to a free particle, and the second term, enclosed in brackets, as the potential the particle feels. This term can be interpreted as an effective potential;
    \[
        V^l_{\text{eff}} (r) \coloneq \frac{\hbar^2}{2\mu} \frac{l(l+1)}{r^2} + V(r).
    \]
    The first term in the effective potential, \(\hbar^2 l(l+1)/2\mu r^2\), can be roughly viewed as arising due to the `centrifugal' force felt by the particle. The eigenvalue equation \eqref{eigen} can be thought of as the equation of a particle in the space \(\rr_{\geq}0\), with the potential \(V^l_{\text{eff}}\). Thus, \(r \in [0, \infty)\).

    The solutions of a differential equation such as \eqref{eigen} depend on the boundary conditions. We have already assumed that \(\hrel\) is self-adjoint on a suitable domain. For our purposes, it is sufficient to restrict our class of potentials which do not increase very rapidly at the origin. More precisely, we require that the the following condition is satisfied.
    \[
        \lim_{r \rightarrow 0} rV(r) = c,
    \]
    where \(c \in \rr\) is a constant. Note that this condition is satisfied for the case we are most interested in, the Coulomb potential. In order to have normalised solutions, we need that the solutions increase rapidly enough at the origin. More precisely, we need the following condition.
    \[
        \lim_{r \rightarrow 0} r^{-(l+1)} u_E(r) = c',
    \]
    where \(c' \in \rr\setminus\{0\}\) is a constant.

    \nocite{*}
    \printbibliography[heading=bibintoc]
    \printindex

\end{document}
