% \DocumentMetadata{pdfstandard=A-3b}
\documentclass[12pt, a4 paper]{article}

\usepackage{embedall}
\embedfile{schuller_two-body_problem.bib}

\usepackage[ngerman,english]{babel}
\PassOptionsToPackage{math-style=ISO, bold-style=ISO, sans-style=italic, nabla=upright, partial=upright, warnings-off={mathtools-colon,mathtools-overbracket}}{unicode-math}
% \PassOptionsToPackage{warnings-off={mathtools-colon,mathtools-overbracket}}{unicode-math}
\usepackage{mathtools, microtype}
\usepackage{setspace}
\usepackage[style=british]{csquotes}
% \usepackage{fontsetup}
% \usepackage[mathscr]{eucal}
% \setmathfontface\itcal{NewCMMath-Book}[FakeSlant=-0.25]
\usepackage{kpfonts-otf}
% \setmainfont[Ligatures=Required]{KpRoman}
% \setmathfont{KpMath-Bold.otf}[version=bold]
% \setmathfont{KpMath-Regular.otf}[range={cal,bfcal},RawFeature=+ss01,BoldFont=KpMath-Bold.otf]
% \setmathfont{KpMath-Regular.otf}[version=regular]
% \setmathfont{KpMath-Regular}[range={cal,bfcal},StylisticSet=1]
% \usepackage{sectsty}
% \allsectionsfont{\mathversion{bold}}
% \setmathfontface\itcal{KpMath-Regular.otf}[FakeSlant=-0.4]
% \usepackage{newtx}
% \usepackage[bold=0.1]{xfakebold}
% \usepackage{xfp}
% \usepackage{dsfont}
% \renewcommand{\mathbb}{\mathbf}
% \renewcommand{\mathbb}[1]{\setBold \mathds #1 \unsetBold}
% \newcommand{\bmathscr}[1]{\setBold \mathscr #1 \unsetBold}
% \renewcommand{\symbb}{\mathds}
% \renewcommand{\symbb}{\symbfup}
% \usepackage[T1]{fontenc}		% use this if European fonts (ec) package is installed
% \usepackage{textcomp}			% adds some additional symbols
% \usepackage[scaled=0.92]{helvet} % set Helvetica as the sans-serif font
% \renewcommand{\rmdefault}{ptm} % set Times as the default text font
% \usepackage{newtxtext}
% \usepackage{fontspec}
% \setmainfont{Times New Roman}
% \setsansfont{Helvetica}
% \setmonofont{Courier}
% \usepackage[lite, subscriptcorrection, compatiblegreek]{mtpro2}

% \usepackage{mlmodern}
\usepackage{amsthm}
\theoremstyle{definition}
\newtheorem{thm}{Theorem}
\newtheorem{defn}{Definition}

\title{The Quantum Mechanical Two-body Problem}
% \author{Frederic Schuller\\Apoorv Potnis}
\author{}
\date{\vspace{-5ex}}

\usepackage[sorting=none]{biblatex}
\addbibresource{schuller_two-body_problem.bib}
\usepackage{hyperref}
\hypersetup{linktoc=all, citecolor=red, pdfencoding=auto, psdextra, colorlinks=true, linkcolor=red, breaklinks=true, urlcolor=blue, pdftitle={Schuller – The Quantum Mechanical Two-body Problem}, bookmarksopen=true, pdfauthor={Apoorv Potnis}, pdfsubject={Two-body Problem – Quantum Mechanics}, unicode=true, pdftoolbar=true, pdfmenubar=true, pdfstartview={FitH}, pdfkeywords={Frederic Schuller, Hydrogen Atom, Quantum Mechanics, Mathematical Physics, Lecture Notes, Angular Momentum, Two-body Problem, Kepler problem}}
% \hypersetup{linktoc=all, citecolor=black, pdfencoding=auto, psdextra, colorlinks=true, linkcolor=black, breaklinks=true, urlcolor=black, pdftitle={Schuller – Rigged Hilbert Spaces}, bookmarksopen=true, pdfauthor={Apoorv Potnis}, pdfsubject={Rigged Hilbert Spaces – Quantum Mechanics}, unicode=true, pdftoolbar=true, pdfmenubar=true, pdfstartview={FitH}, pdfkeywords={Frederic Schuller, Rigged Hilbert Spaces, Quantum Mechanics, Mathematical Physics, Lecture Notes, Bras, Kets}}
\usepackage{cleveref, xurl, bookmark}

% % \newcommand{\ltwo}{\mathup{L\kern-0.5pt^2}}
% \newcommand{\position}{\mathup{Q}}
% \newcommand{\momentum}{\mathup{P}}
% \newcommand{\rthree}{\mathbb{R}^3}
% \newcommand{\rr}{\mathbb{R}}
% \newcommand{\cc}{\mathbb{C}}
% \newcommand{\nn}{\mathbb{N}_0}
% \newcommand{\dirac}{\symup{\delta}}
% % \newcommand{\hilbert}{\itcal{ℋ}\kern-0.7pt}
% \newcommand{\hilbert}{\mathcal{H}}
% \newcommand{\ltwor}{\ltwo(\rr)}
% % \newcommand{\poly}{\itcal{𝒫}\kern-0.5pt}
% \newcommand{\poly}{𝒫}
% \newcommand{\schwartz}{\kern0.5pt\itcal{𝒮}\kern-2pt}
% % \newcommand{\schwartz}{\mathcal{S}}
% \newcommand{\schwartzr}{\schwartz(\rr)}
% \newcommand{\dist}{\itcal{\schwartz^\times}\kern-0.5pt}
% % \newcommand{\dist}{\schwartz^\times}
% \newcommand{\distr}{\dist(\rr)}
% \newcommand{\distar}{\schwartz^{\kern-0.1pt*}(\rr)}
% \newcommand{\anti}{\overline{\distr}}
% \renewcommand*{\hbar}{\mathrm{^^^^0127}}
% \renewcommand{\i}{\mathrm{i}}
% \newcommand{\e}{\mathrm{e}}
% \newcommand{\cinfinity}{\mathrm{C}^\infty}
% \newcommand{\family}{\mathcal{F}}
% \newcommand{\lift}{\hat{D}}
% \newcommand{\domain}{\mathcal{D}}
% \newcommand{\identity}{\mathrm{id}}
% \newcommand{\w}{w\kern-1pt}
% \DeclareMathOperator{\spec}{spec}
% \newcommand{\resolvent}{\mathup{\rho}}

\newcommand{\ltwo}{\mathup{L\kern-0.5pt^2}}
\newcommand{\position}{Q}
\newcommand{\momentum}{P}
\newcommand{\rthree}{\mathbb{R}^3}
\newcommand{\rr}{\mathbb{R}}
\newcommand{\cc}{\mathbb{C}}
\newcommand{\nn}{\mathbb{N}_0}
\newcommand{\dirac}{\delta}
% \newcommand{\hilbert}{\itcal{ℋ}\kern-0.7pt}
\newcommand{\hilbert}{\mathcal{H}}
\newcommand{\ltwor}{\ltwo(\rr)}
% \newcommand{\poly}{\itcal{𝒫}\kern-0.5pt}
\newcommand{\poly}{\mathcal{P}}
\newcommand{\schwartz}{\mathcal{S}}
% \newcommand{\schwartz}{\mathcal{S}}
\newcommand{\schwartzr}{\schwartz(\rr)}
\newcommand{\dist}{\schwartz^\times}
% \newcommand{\dist}{\schwartz^\times}
\newcommand{\distr}{\dist(\rr)}
\newcommand{\distar}{\schwartz^*(\rr)}
\newcommand{\anti}{\overline{\distr}}
\renewcommand*{\hbar}{\mathrm{^^^^0127}}
\renewcommand{\i}{\mathrm{i}}
\newcommand{\e}{\mathrm{e}}
\newcommand{\cinfinity}{\mathrm{C}^\infty}
\newcommand{\family}{\mathcal{F}}
\newcommand{\lift}{\hat{D}}
\newcommand{\domain}{\mathcal{D}}
\newcommand{\identity}{\mathrm{id}}
\newcommand{\w}{w\kern-1pt}
\DeclareMathOperator{\spec}{spec}
\newcommand{\resolvent}{\mathup{\rho}}

\DeclarePairedDelimiter{\norm}{\lVert}{\rVert}
\DeclarePairedDelimiter{\abs}{\lvert}{\rvert}
\newcommand{\der}{\operatorname{d\!}{}}
\usepackage{mleftright}

\usepackage{luatex85}
\newsavebox{\foobox}
\newcommand{\slantbox}[2][.5]
{%
	\mbox
	{%
		\sbox{\foobox}{#2}%
		\hskip\wd\foobox
		\pdfsave
		\pdfsetmatrix{1 0 #1 1}%
		\llap{\usebox{\foobox}}%
		\pdfrestore
	}%
}

\usepackage{etoolbox}
\apptocmd{\sloppy}{\hbadness 10000\relax}{}{}
\emergencystretch=1em

\makeatletter
\g@addto@macro\bfseries{\boldmath}
\makeatother

\usepackage{tikz, tkz-euclide, pgfplots}
\pgfplotsset{compat=1.18}
\setmathfont{KpMath-Sans.otf}[version=sans]
\setmathfont{KpMath-Regular.otf}[version=base]
\usepackage{tikz-cd}
\tikzcdset{arrow style=math font}
\usetikzlibrary{decorations.markings}
\usetikzlibrary {arrows.meta,bending}
% \setmathfont{KpMath-Regular_arrow.otf}
% \tikzset{
% 	-math/.style={
% 		shorten >=1.4\pgflinewidth,
% 		postaction=decorate,
% 		decoration={markings, mark=at position 0.89 with {
% 				\node[scale=1.4\pgflinewidth, xshift=-1.7pt,transform shape,rotate=-91]{$\char"2191$}; }}}
% }
% \usepackage{fontsetup}
% \setmathfont{KpMath-Regular_arrow.otf}
\begin{document}

	\maketitle

	These are lecture notes by Apoorv Potnis of the lecture `\selectlanguage{ngerman}Quantenmechanisches Zweikörperproblem\selectlanguage{english}' or `The Quantum Mechanical Two-body Problem', given by \textbf{Prof.\@ Frederic Paul Schuller}, as the seventeenth lecture in the course `\selectlanguage{ngerman}Theoretische Physik 2: Theoretische Quantenmechanik\selectlanguage{english}' in 2014/15 at the \selectlanguage{ngerman}Friedrich-Alexander-Universität Erlangen-Nürnberg\selectlanguage{english}. While the original lecture is in German, these notes are in English and have been prepared using YouTube's automatic subtitle translation tool. The lecture is available at \url{https://www.youtube.com/watch?v=mcM4S3IMMvI&list=PLPO5pgr_frzTeqa_thbltYjyw8F9ehw7v&index=17} and at \url{https://www.fau.tv/clip/id/44891}.

	\tableofcontents

	\section{Introduction}

	\begin{figure}
		\mathversion{sans}
	    \centering
		\begin{tikzpicture}
% 			\node (mone) at (0,0) {\(m_1\)};
			\filldraw [black] (0,0) circle (6pt);
			\node (mone) at (0.4,0.4) {\(m_1\)};
			\filldraw [black] (1.6,-1.6) circle (6pt);
			\node (mtwo) at (1.2,-2) {\(m_2\)};
			\draw [black, dashed, thick] (0,0) -- (1.6,-1.6) node [midway, above right] {\(r\)};
			\tkzDefPoint(0.8,-0.8){O}
			\tkzDrawArc[R, color=black, line width=0.6pt, -{Straight Barb[bend]}](O,1.1cm)(340,10)
			\tkzDrawArc[R, color=black, line width=0.6pt, -{Straight Barb[bend]}](O,1.1cm)(160,190)

% 			\draw [black, thick, ->] (0,0) arc [start angle=10, end angle=40, x radius=1.6cm, y radius =1.6cm];

% 			\draw[black, thick, ->, xshift=0.8cm, yshift=-0.8cm] (-30:0.8cm) arc (-30:-10:0.8cm);
		\end{tikzpicture}
		\mathversion{base}
	\end{figure}
	In this lecture, we shall consider a quantum-mechanical system consisting of two interacting particles of masses \(m_1\) and \(m_2\), such that the interaction is completely determined by the potential $V(r)$, and the potential depends only on the distance $r$ between the particles. An example of such a potential would be the \textit{Yukawa potential}, defined as
	\begin{align*}
		V_{\text{Yukawa}}(r) \coloneq a \frac{\exp({-kmr})}{r},
	\end{align*}
	where $k$, $m$ and $a$ are constants. $a \in \rr \setminus \{0\}$, $m \geq 0$. According to quantum field theory, very roughly speaking, interaction between particles takes place via a `mediating particle'. If the interaction is mediated by a `scalar field'\footnote{Whatever that means}, then the mass associated to the particle of that scalar field is the mass $m$ appearing in the Yukawa potential. If we plot a graph of the Yukawa potential for a massive scalar field, then we see that the magnitude of the potential becomes very close to zero after a certain distance. Thus, these interactions are short-ranged. If instead we have $m = 0$, corresponding to a photon, then we get the familiar long-range Coulomb potential
	\begin{align*}
		V_{\text{Coulomb}}(r) \coloneq a \frac{1}{r}.
	\end{align*}
	\begin{figure}
	    \centering
		\mathversion{sans}
		\begin{tikzpicture}[line cap=round]
			%		%Grid
			%		\draw[thin, dotted] (0,0) grid (8,8);
			%		\foreach \i in {1,...,8}
			%		{
			%			\node at (\i,-2ex) {\i};
			%		}
			%		\foreach \i in {1,...,8}
			%		{
			%			\node at (-2ex,\i) {\i};
			%		}
			%		\node at (-2ex,-2ex) {0};

% 			% Axis
% 			\draw[thick, -{Straight Barb[bend]}] (0,0) -- (5,0) node[above] {$r$};
% 			\draw[thick, {Straight Barb[bend]}-{Straight Barb[bend]}] (0,-4.5) -- (0,0.5) node[right] {$V(r)$};
% 			\node at (-0.3,0) {$0$};

			\begin{axis}[
				samples=500,
				domain=0.24:4.5,
				xmin=0, xmax=5,
				ymin=-4.4, ymax=0.5,
				axis lines=middle,
				ticks=none,
				width=0.5\textwidth,
% 				height=5cm,
				xlabel={$r$},
				ylabel={$V(r)$},
				x axis line style={thick, -{Straight Barb[bend]}},
				y axis line style={thick, {Straight Barb[bend]}-{Straight Barb[bend]}},
				y label style={anchor=south},
				x label style={anchor=west},
				legend style={draw=none, at={(axis cs:2,-1.8)},anchor=north west},
				legend style={row sep=10pt}
				]
				\addplot[thick, red] {-exp(-\x)/(\x)};
				\addplot[thick, blue, dashed] {-1/(\x)};
				\addlegendentry{$\displaystyle-\frac{e^{-r}}{r}$};
				\addlegendentry{$\displaystyle-\frac{1}{r}$};
			\end{axis};


			% Plot Function
	% 		\draw[domain=0.177:7.5, samples=600, variable=\r, very thick] plot ({\r},{0.3/(\r*\r)-0.9/\r});
	% 		\draw[domain=0.6:7.5, samples=500, variable=\r, thick, dashed, red] plot ({\r},{1/(\r*\r)});
% 			\draw[domain=0.24:4.5, samples=500, variable=\r, thick, blue] plot ({\r},{-1/(\r)});
% 			\draw[domain=0.24:4.5, samples=500, variable=\r, thick, red] plot ({\r},{-exp(-\r)/(\r)});

			% Dashed
% 			\draw[dashed] (2/3,0) -- +(0,-0.65) node[pos=0, above] {$r_o$};
% 			\draw[dashed] (2/3,-0.68) -- (0,-0.68);

			% Nodes
% 			\node at (0.6,3.5) {$V_\text{eff}$};
% 			\node[red] at (1.3,1.5) {$1/r^2$};
% 			\node[blue] at (1.4,-1.5) {$-1/r$};
		\end{tikzpicture}
		\mathversion{base}
	\end{figure}




%

	\nocite{*}
	\printbibliography[heading=bibintoc]

	\par\begin{spacing}{0.6}
		{\footnotesize The source code, updates and corrections to this document can be found on this GitHub repository: \url{https://github.com/apoorvpotnis/schuller_two-body_problem}. The source code, along with the \texttt{.bib} file is embedded in this PDF. Comments and corrections can be mailed at \href{mailto:apoorvpotnis@gmail.com}{\texttt{apoorvpotnis@gmail.com}}.}
	\end{spacing}

\end{document}
